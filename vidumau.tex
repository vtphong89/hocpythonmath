
\begin{ex}
 Trên mặt phẳng tọa độ, cho $M(-2,-3)$ là điểm biểu diển của số phức $z$. Phần thực của số phức $z$ là
\choice
{\True $-2$}
{$2$}
{$3$}
{$-3$}
\loigiai{Noi dung loi giai.
}
\end{ex}
\begin{ex}
 Trên mặt phẳng tọa độ, cho $M(1,2)$ là điểm biểu diển của số phức $z$. Phần thực của số phức $z$ là
\choice
{$-2$}
{$2$}
{\True $1$}
{$-1$}
\loigiai{Noi dung loi giai.
}
\end{ex}
\begin{ex}
 Trên mặt phẳng tọa độ, cho $M(2,-1)$ là điểm biểu diển của số phức $z$. Phần thực của số phức $z$ là
\choice
{\True $2$}
{$-1$}
{$-2$}
{$1$}
\loigiai{Noi dung loi giai.
}
\end{ex}
\begin{ex}
 Trên mặt phẳng tọa độ, cho $M(2,1)$ là điểm biểu diển của số phức $z$. Phần thực của số phức $z$ là
\choice
{$1$}
{\True $2$}
{$-2$}
{$-1$}
\loigiai{Noi dung loi giai.
}
\end{ex}
\begin{ex}
 Trên mặt phẳng tọa độ, cho $M(2,1)$ là điểm biểu diển của số phức $z$. Phần thực của số phức $z$ là
\choice
{$-1$}
{\True $2$}
{$1$}
{$-2$}
\loigiai{Noi dung loi giai.
}
\end{ex}
\begin{ex}
 Trên mặt phẳng tọa độ, cho $M(2,-1)$ là điểm biểu diển của số phức $z$. Phần thực của số phức $z$ là
\choice
{$-1$}
{$1$}
{$-2$}
{\True $2$}
\loigiai{Noi dung loi giai.
}
\end{ex}
\begin{ex}
 Trên mặt phẳng tọa độ, cho $M(1,2)$ là điểm biểu diển của số phức $z$. Phần thực của số phức $z$ là
\choice
{$2$}
{\True $1$}
{$-1$}
{$-2$}
\loigiai{Noi dung loi giai.
}
\end{ex}
\begin{ex}
 Trên mặt phẳng tọa độ, cho $M(-3,1)$ là điểm biểu diển của số phức $z$. Phần thực của số phức $z$ là
\choice
{$-1$}
{$1$}
{\True $-3$}
{$3$}
\loigiai{Noi dung loi giai.
}
\end{ex}
\begin{ex}
 Trên mặt phẳng tọa độ, cho $M(2,-3)$ là điểm biểu diển của số phức $z$. Phần thực của số phức $z$ là
\choice
{\True $2$}
{$-3$}
{$3$}
{$-2$}
\loigiai{Noi dung loi giai.
}
\end{ex}
\begin{ex}
 Trên mặt phẳng tọa độ, cho $M(-2,-3)$ là điểm biểu diển của số phức $z$. Phần thực của số phức $z$ là
\choice
{\True $-2$}
{$2$}
{$3$}
{$-3$}
\loigiai{Noi dung loi giai.
}
\end{ex}
\begin{ex}
 Trên mặt phẳng tọa độ, cho $M(-1,-2)$ là điểm biểu diển của số phức $z$. Phần thực của số phức $z$ là
\choice
{$-2$}
{$2$}
{$1$}
{\True $-1$}
\loigiai{Noi dung loi giai.
}
\end{ex}
\begin{ex}
 Trên mặt phẳng tọa độ, cho $M(-3,-2)$ là điểm biểu diển của số phức $z$. Phần thực của số phức $z$ là
\choice
{$-2$}
{$2$}
{\True $-3$}
{$3$}
\loigiai{Noi dung loi giai.
}
\end{ex}
\begin{ex}
 Trên mặt phẳng tọa độ, cho $M(-2,-3)$ là điểm biểu diển của số phức $z$. Phần thực của số phức $z$ là
\choice
{$-3$}
{$3$}
{\True $-2$}
{$2$}
\loigiai{Noi dung loi giai.
}
\end{ex}
\begin{ex}
 Trên mặt phẳng tọa độ, cho $M(1,2)$ là điểm biểu diển của số phức $z$. Phần thực của số phức $z$ là
\choice
{$-1$}
{$-2$}
{$2$}
{\True $1$}
\loigiai{Noi dung loi giai.
}
\end{ex}
\begin{ex}
 Trên mặt phẳng tọa độ, cho $M(2,-1)$ là điểm biểu diển của số phức $z$. Phần thực của số phức $z$ là
\choice
{$-2$}
{$1$}
{$-1$}
{\True $2$}
\loigiai{Noi dung loi giai.
}
\end{ex}
\begin{ex}
 Trên mặt phẳng tọa độ, cho $M(-2,-3)$ là điểm biểu diển của số phức $z$. Phần thực của số phức $z$ là
\choice
{\True $-2$}
{$3$}
{$-3$}
{$2$}
\loigiai{Noi dung loi giai.
}
\end{ex}
\begin{ex}
 Trên mặt phẳng tọa độ, cho $M(1,-3)$ là điểm biểu diển của số phức $z$. Phần thực của số phức $z$ là
\choice
{$-3$}
{$3$}
{\True $1$}
{$-1$}
\loigiai{Noi dung loi giai.
}
\end{ex}
\begin{ex}
 Trên mặt phẳng tọa độ, cho $M(-1,-3)$ là điểm biểu diển của số phức $z$. Phần thực của số phức $z$ là
\choice
{$-3$}
{$1$}
{$3$}
{\True $-1$}
\loigiai{Noi dung loi giai.
}
\end{ex}
\begin{ex}
 Trên mặt phẳng tọa độ, cho $M(-1,-3)$ là điểm biểu diển của số phức $z$. Phần thực của số phức $z$ là
\choice
{$3$}
{\True $-1$}
{$-3$}
{$1$}
\loigiai{Noi dung loi giai.
}
\end{ex}
\begin{ex}
 Trên mặt phẳng tọa độ, cho $M(2,-1)$ là điểm biểu diển của số phức $z$. Phần thực của số phức $z$ là
\choice
{$1$}
{$-2$}
{$-1$}
{\True $2$}
\loigiai{Noi dung loi giai.
}
\end{ex}
\begin{ex}
 Trên mặt phẳng tọa độ, cho $M(2,1)$ là điểm biểu diển của số phức $z$. Phần thực của số phức $z$ là
\choice
{$-2$}
{$-1$}
{\True $2$}
{$1$}
\loigiai{Noi dung loi giai.
}
\end{ex}
\begin{ex}
 Trên mặt phẳng tọa độ, cho $M(1,-2)$ là điểm biểu diển của số phức $z$. Phần thực của số phức $z$ là
\choice
{\True $1$}
{$-1$}
{$2$}
{$-2$}
\loigiai{Noi dung loi giai.
}
\end{ex}
\begin{ex}
 Trên mặt phẳng tọa độ, cho $M(-3,-2)$ là điểm biểu diển của số phức $z$. Phần thực của số phức $z$ là
\choice
{$-2$}
{$2$}
{\True $-3$}
{$3$}
\loigiai{Noi dung loi giai.
}
\end{ex}
\begin{ex}
 Trên mặt phẳng tọa độ, cho $M(-3,-1)$ là điểm biểu diển của số phức $z$. Phần thực của số phức $z$ là
\choice
{\True $-3$}
{$-1$}
{$3$}
{$1$}
\loigiai{Noi dung loi giai.
}
\end{ex}
\begin{ex}
 Trên mặt phẳng tọa độ, cho $M(-3,1)$ là điểm biểu diển của số phức $z$. Phần thực của số phức $z$ là
\choice
{$1$}
{\True $-3$}
{$3$}
{$-1$}
\loigiai{Noi dung loi giai.
}
\end{ex}
\begin{ex}
 Trên mặt phẳng tọa độ, cho $M(-2,-1)$ là điểm biểu diển của số phức $z$. Phần thực của số phức $z$ là
\choice
{$-1$}
{$2$}
{\True $-2$}
{$1$}
\loigiai{Noi dung loi giai.
}
\end{ex}
\begin{ex}
 Trên mặt phẳng tọa độ, cho $M(2,-1)$ là điểm biểu diển của số phức $z$. Phần thực của số phức $z$ là
\choice
{$1$}
{$-1$}
{$-2$}
{\True $2$}
\loigiai{Noi dung loi giai.
}
\end{ex}
\begin{ex}
 Trên mặt phẳng tọa độ, cho $M(1,-2)$ là điểm biểu diển của số phức $z$. Phần thực của số phức $z$ là
\choice
{\True $1$}
{$-1$}
{$2$}
{$-2$}
\loigiai{Noi dung loi giai.
}
\end{ex}
\begin{ex}
 Trên mặt phẳng tọa độ, cho $M(-3,2)$ là điểm biểu diển của số phức $z$. Phần thực của số phức $z$ là
\choice
{$2$}
{$-2$}
{\True $-3$}
{$3$}
\loigiai{Noi dung loi giai.
}
\end{ex}
\begin{ex}
 Trên mặt phẳng tọa độ, cho $M(-1,2)$ là điểm biểu diển của số phức $z$. Phần thực của số phức $z$ là
\choice
{$1$}
{\True $-1$}
{$2$}
{$-2$}
\loigiai{Noi dung loi giai.
}
\end{ex}
\begin{ex}
 Trên mặt phẳng tọa độ, cho $M(-2,1)$ là điểm biểu diển của số phức $z$. Phần thực của số phức $z$ là
\choice
{\True $-2$}
{$1$}
{$2$}
{$-1$}
\loigiai{Noi dung loi giai.
}
\end{ex}
\begin{ex}
 Trên mặt phẳng tọa độ, cho $M(1,2)$ là điểm biểu diển của số phức $z$. Phần thực của số phức $z$ là
\choice
{$-2$}
{$-1$}
{$2$}
{\True $1$}
\loigiai{Noi dung loi giai.
}
\end{ex}
\begin{ex}
 Trên mặt phẳng tọa độ, cho $M(1,2)$ là điểm biểu diển của số phức $z$. Phần thực của số phức $z$ là
\choice
{$-2$}
{$-1$}
{\True $1$}
{$2$}
\loigiai{Noi dung loi giai.
}
\end{ex}
\begin{ex}
 Trên mặt phẳng tọa độ, cho $M(1,2)$ là điểm biểu diển của số phức $z$. Phần thực của số phức $z$ là
\choice
{\True $1$}
{$-1$}
{$-2$}
{$2$}
\loigiai{Noi dung loi giai.
}
\end{ex}
\begin{ex}
 Trên mặt phẳng tọa độ, cho $M(-3,2)$ là điểm biểu diển của số phức $z$. Phần thực của số phức $z$ là
\choice
{$2$}
{$3$}
{$-2$}
{\True $-3$}
\loigiai{Noi dung loi giai.
}
\end{ex}
\begin{ex}
 Trên mặt phẳng tọa độ, cho $M(-3,2)$ là điểm biểu diển của số phức $z$. Phần thực của số phức $z$ là
\choice
{\True $-3$}
{$3$}
{$2$}
{$-2$}
\loigiai{Noi dung loi giai.
}
\end{ex}
\begin{ex}
 Trên mặt phẳng tọa độ, cho $M(-3,1)$ là điểm biểu diển của số phức $z$. Phần thực của số phức $z$ là
\choice
{$3$}
{\True $-3$}
{$1$}
{$-1$}
\loigiai{Noi dung loi giai.
}
\end{ex}
\begin{ex}
 Trên mặt phẳng tọa độ, cho $M(1,2)$ là điểm biểu diển của số phức $z$. Phần thực của số phức $z$ là
\choice
{$-1$}
{\True $1$}
{$2$}
{$-2$}
\loigiai{Noi dung loi giai.
}
\end{ex}
\begin{ex}
 Trên mặt phẳng tọa độ, cho $M(-1,-2)$ là điểm biểu diển của số phức $z$. Phần thực của số phức $z$ là
\choice
{$-2$}
{\True $-1$}
{$1$}
{$2$}
\loigiai{Noi dung loi giai.
}
\end{ex}
\begin{ex}
 Trên mặt phẳng tọa độ, cho $M(2,-3)$ là điểm biểu diển của số phức $z$. Phần thực của số phức $z$ là
\choice
{$-2$}
{\True $2$}
{$-3$}
{$3$}
\loigiai{Noi dung loi giai.
}
\end{ex}
\begin{ex}
 Trên mặt phẳng tọa độ, cho $M(1,-3)$ là điểm biểu diển của số phức $z$. Phần thực của số phức $z$ là
\choice
{$-1$}
{$-3$}
{\True $1$}
{$3$}
\loigiai{Noi dung loi giai.
}
\end{ex}
\begin{ex}
 Trên mặt phẳng tọa độ, cho $M(2,1)$ là điểm biểu diển của số phức $z$. Phần thực của số phức $z$ là
\choice
{$1$}
{$-2$}
{$-1$}
{\True $2$}
\loigiai{Noi dung loi giai.
}
\end{ex}
\begin{ex}
 Trên mặt phẳng tọa độ, cho $M(2,-3)$ là điểm biểu diển của số phức $z$. Phần thực của số phức $z$ là
\choice
{\True $2$}
{$-2$}
{$-3$}
{$3$}
\loigiai{Noi dung loi giai.
}
\end{ex}
\begin{ex}
 Trên mặt phẳng tọa độ, cho $M(-1,-3)$ là điểm biểu diển của số phức $z$. Phần thực của số phức $z$ là
\choice
{$3$}
{\True $-1$}
{$-3$}
{$1$}
\loigiai{Noi dung loi giai.
}
\end{ex}
\begin{ex}
 Trên mặt phẳng tọa độ, cho $M(1,-3)$ là điểm biểu diển của số phức $z$. Phần thực của số phức $z$ là
\choice
{$-1$}
{$3$}
{$-3$}
{\True $1$}
\loigiai{Noi dung loi giai.
}
\end{ex}
\begin{ex}
 Trên mặt phẳng tọa độ, cho $M(-3,-2)$ là điểm biểu diển của số phức $z$. Phần thực của số phức $z$ là
\choice
{$3$}
{$2$}
{\True $-3$}
{$-2$}
\loigiai{Noi dung loi giai.
}
\end{ex}
\begin{ex}
 Trên mặt phẳng tọa độ, cho $M(-3,2)$ là điểm biểu diển của số phức $z$. Phần thực của số phức $z$ là
\choice
{\True $-3$}
{$3$}
{$-2$}
{$2$}
\loigiai{Noi dung loi giai.
}
\end{ex}
\begin{ex}
 Trên mặt phẳng tọa độ, cho $M(-1,2)$ là điểm biểu diển của số phức $z$. Phần thực của số phức $z$ là
\choice
{$-2$}
{\True $-1$}
{$2$}
{$1$}
\loigiai{Noi dung loi giai.
}
\end{ex}
\begin{ex}
 Trên mặt phẳng tọa độ, cho $M(-2,-1)$ là điểm biểu diển của số phức $z$. Phần thực của số phức $z$ là
\choice
{$-1$}
{$1$}
{$2$}
{\True $-2$}
\loigiai{Noi dung loi giai.
}
\end{ex}
\begin{ex}
 Trên mặt phẳng tọa độ, cho $M(-3,2)$ là điểm biểu diển của số phức $z$. Phần thực của số phức $z$ là
\choice
{$3$}
{$-2$}
{\True $-3$}
{$2$}
\loigiai{Noi dung loi giai.
}
\end{ex}
\begin{ex}
 Trên mặt phẳng tọa độ, cho $M(-3,-2)$ là điểm biểu diển của số phức $z$. Phần thực của số phức $z$ là
\choice
{$-2$}
{$2$}
{\True $-3$}
{$3$}
\loigiai{Noi dung loi giai.
}
\end{ex}
\begin{ex}
 Trên mặt phẳng tọa độ, cho $M(1,-2)$ là điểm biểu diển của số phức $z$. Phần thực của số phức $z$ là
\choice
{\True $1$}
{$2$}
{$-2$}
{$-1$}
\loigiai{Noi dung loi giai.
}
\end{ex}
\begin{ex}
 Trên mặt phẳng tọa độ, cho $M(-1,-3)$ là điểm biểu diển của số phức $z$. Phần thực của số phức $z$ là
\choice
{\True $-1$}
{$3$}
{$1$}
{$-3$}
\loigiai{Noi dung loi giai.
}
\end{ex}
\begin{ex}
 Trên mặt phẳng tọa độ, cho $M(-2,-3)$ là điểm biểu diển của số phức $z$. Phần thực của số phức $z$ là
\choice
{$3$}
{$-3$}
{\True $-2$}
{$2$}
\loigiai{Noi dung loi giai.
}
\end{ex}
\begin{ex}
 Trên mặt phẳng tọa độ, cho $M(-2,1)$ là điểm biểu diển của số phức $z$. Phần thực của số phức $z$ là
\choice
{\True $-2$}
{$-1$}
{$2$}
{$1$}
\loigiai{Noi dung loi giai.
}
\end{ex}
\begin{ex}
 Trên mặt phẳng tọa độ, cho $M(-2,-3)$ là điểm biểu diển của số phức $z$. Phần thực của số phức $z$ là
\choice
{\True $-2$}
{$-3$}
{$2$}
{$3$}
\loigiai{Noi dung loi giai.
}
\end{ex}
\begin{ex}
 Trên mặt phẳng tọa độ, cho $M(-3,2)$ là điểm biểu diển của số phức $z$. Phần thực của số phức $z$ là
\choice
{$3$}
{$-2$}
{$2$}
{\True $-3$}
\loigiai{Noi dung loi giai.
}
\end{ex}
\begin{ex}
 Trên mặt phẳng tọa độ, cho $M(1,-2)$ là điểm biểu diển của số phức $z$. Phần thực của số phức $z$ là
\choice
{$-1$}
{\True $1$}
{$2$}
{$-2$}
\loigiai{Noi dung loi giai.
}
\end{ex}
\begin{ex}
 Trên mặt phẳng tọa độ, cho $M(-3,-2)$ là điểm biểu diển của số phức $z$. Phần thực của số phức $z$ là
\choice
{$-2$}
{\True $-3$}
{$2$}
{$3$}
\loigiai{Noi dung loi giai.
}
\end{ex}
\begin{ex}
 Trên mặt phẳng tọa độ, cho $M(-3,-2)$ là điểm biểu diển của số phức $z$. Phần thực của số phức $z$ là
\choice
{$3$}
{\True $-3$}
{$-2$}
{$2$}
\loigiai{Noi dung loi giai.
}
\end{ex}
\begin{ex}
 Trên mặt phẳng tọa độ, cho $M(2,-3)$ là điểm biểu diển của số phức $z$. Phần thực của số phức $z$ là
\choice
{\True $2$}
{$-3$}
{$-2$}
{$3$}
\loigiai{Noi dung loi giai.
}
\end{ex}
\begin{ex}
 Trên mặt phẳng tọa độ, cho $M(-1,2)$ là điểm biểu diển của số phức $z$. Phần thực của số phức $z$ là
\choice
{$-2$}
{\True $-1$}
{$2$}
{$1$}
\loigiai{Noi dung loi giai.
}
\end{ex}
\begin{ex}
 Trên mặt phẳng tọa độ, cho $M(-3,1)$ là điểm biểu diển của số phức $z$. Phần thực của số phức $z$ là
\choice
{$-1$}
{$3$}
{\True $-3$}
{$1$}
\loigiai{Noi dung loi giai.
}
\end{ex}
\begin{ex}
 Trên mặt phẳng tọa độ, cho $M(-2,-1)$ là điểm biểu diển của số phức $z$. Phần thực của số phức $z$ là
\choice
{$1$}
{$2$}
{\True $-2$}
{$-1$}
\loigiai{Noi dung loi giai.
}
\end{ex}
\begin{ex}
 Trên mặt phẳng tọa độ, cho $M(-1,-2)$ là điểm biểu diển của số phức $z$. Phần thực của số phức $z$ là
\choice
{$-2$}
{$1$}
{\True $-1$}
{$2$}
\loigiai{Noi dung loi giai.
}
\end{ex}
\begin{ex}
 Trên mặt phẳng tọa độ, cho $M(2,-1)$ là điểm biểu diển của số phức $z$. Phần thực của số phức $z$ là
\choice
{$1$}
{\True $2$}
{$-2$}
{$-1$}
\loigiai{Noi dung loi giai.
}
\end{ex}
\begin{ex}
 Trên mặt phẳng tọa độ, cho $M(-3,2)$ là điểm biểu diển của số phức $z$. Phần thực của số phức $z$ là
\choice
{$-2$}
{$3$}
{$2$}
{\True $-3$}
\loigiai{Noi dung loi giai.
}
\end{ex}
\begin{ex}
 Trên mặt phẳng tọa độ, cho $M(-2,1)$ là điểm biểu diển của số phức $z$. Phần thực của số phức $z$ là
\choice
{$-1$}
{$1$}
{\True $-2$}
{$2$}
\loigiai{Noi dung loi giai.
}
\end{ex}
\begin{ex}
 Trên mặt phẳng tọa độ, cho $M(-1,-3)$ là điểm biểu diển của số phức $z$. Phần thực của số phức $z$ là
\choice
{$-3$}
{$3$}
{$1$}
{\True $-1$}
\loigiai{Noi dung loi giai.
}
\end{ex}
\begin{ex}
 Trên mặt phẳng tọa độ, cho $M(1,-2)$ là điểm biểu diển của số phức $z$. Phần thực của số phức $z$ là
\choice
{\True $1$}
{$-2$}
{$2$}
{$-1$}
\loigiai{Noi dung loi giai.
}
\end{ex}
\begin{ex}
 Trên mặt phẳng tọa độ, cho $M(2,-1)$ là điểm biểu diển của số phức $z$. Phần thực của số phức $z$ là
\choice
{$1$}
{$-2$}
{\True $2$}
{$-1$}
\loigiai{Noi dung loi giai.
}
\end{ex}
\begin{ex}
 Trên mặt phẳng tọa độ, cho $M(-2,-1)$ là điểm biểu diển của số phức $z$. Phần thực của số phức $z$ là
\choice
{\True $-2$}
{$2$}
{$-1$}
{$1$}
\loigiai{Noi dung loi giai.
}
\end{ex}
\begin{ex}
 Trên mặt phẳng tọa độ, cho $M(-1,2)$ là điểm biểu diển của số phức $z$. Phần thực của số phức $z$ là
\choice
{$2$}
{\True $-1$}
{$-2$}
{$1$}
\loigiai{Noi dung loi giai.
}
\end{ex}
\begin{ex}
 Trên mặt phẳng tọa độ, cho $M(-1,-2)$ là điểm biểu diển của số phức $z$. Phần thực của số phức $z$ là
\choice
{$-2$}
{$1$}
{\True $-1$}
{$2$}
\loigiai{Noi dung loi giai.
}
\end{ex}
\begin{ex}
 Trên mặt phẳng tọa độ, cho $M(2,1)$ là điểm biểu diển của số phức $z$. Phần thực của số phức $z$ là
\choice
{$1$}
{$-1$}
{\True $2$}
{$-2$}
\loigiai{Noi dung loi giai.
}
\end{ex}
\begin{ex}
 Trên mặt phẳng tọa độ, cho $M(1,-2)$ là điểm biểu diển của số phức $z$. Phần thực của số phức $z$ là
\choice
{$-2$}
{$2$}
{$-1$}
{\True $1$}
\loigiai{Noi dung loi giai.
}
\end{ex}
\begin{ex}
 Trên mặt phẳng tọa độ, cho $M(1,-3)$ là điểm biểu diển của số phức $z$. Phần thực của số phức $z$ là
\choice
{$3$}
{$-3$}
{$-1$}
{\True $1$}
\loigiai{Noi dung loi giai.
}
\end{ex}
\begin{ex}
 Trên mặt phẳng tọa độ, cho $M(-1,-2)$ là điểm biểu diển của số phức $z$. Phần thực của số phức $z$ là
\choice
{$2$}
{$1$}
{\True $-1$}
{$-2$}
\loigiai{Noi dung loi giai.
}
\end{ex}
\begin{ex}
 Trên mặt phẳng tọa độ, cho $M(-3,-2)$ là điểm biểu diển của số phức $z$. Phần thực của số phức $z$ là
\choice
{\True $-3$}
{$3$}
{$-2$}
{$2$}
\loigiai{Noi dung loi giai.
}
\end{ex}
\begin{ex}
 Trên mặt phẳng tọa độ, cho $M(-1,-2)$ là điểm biểu diển của số phức $z$. Phần thực của số phức $z$ là
\choice
{$1$}
{$2$}
{\True $-1$}
{$-2$}
\loigiai{Noi dung loi giai.
}
\end{ex}
\begin{ex}
 Trên mặt phẳng tọa độ, cho $M(-2,-3)$ là điểm biểu diển của số phức $z$. Phần thực của số phức $z$ là
\choice
{$3$}
{$2$}
{\True $-2$}
{$-3$}
\loigiai{Noi dung loi giai.
}
\end{ex}
\begin{ex}
 Trên mặt phẳng tọa độ, cho $M(1,2)$ là điểm biểu diển của số phức $z$. Phần thực của số phức $z$ là
\choice
{$-1$}
{\True $1$}
{$2$}
{$-2$}
\loigiai{Noi dung loi giai.
}
\end{ex}
\begin{ex}
 Trên mặt phẳng tọa độ, cho $M(2,-3)$ là điểm biểu diển của số phức $z$. Phần thực của số phức $z$ là
\choice
{\True $2$}
{$-3$}
{$3$}
{$-2$}
\loigiai{Noi dung loi giai.
}
\end{ex}
\begin{ex}
 Trên mặt phẳng tọa độ, cho $M(-2,-3)$ là điểm biểu diển của số phức $z$. Phần thực của số phức $z$ là
\choice
{$3$}
{\True $-2$}
{$2$}
{$-3$}
\loigiai{Noi dung loi giai.
}
\end{ex}
\begin{ex}
 Trên mặt phẳng tọa độ, cho $M(1,-2)$ là điểm biểu diển của số phức $z$. Phần thực của số phức $z$ là
\choice
{$-1$}
{$-2$}
{\True $1$}
{$2$}
\loigiai{Noi dung loi giai.
}
\end{ex}
\begin{ex}
 Trên mặt phẳng tọa độ, cho $M(-3,-2)$ là điểm biểu diển của số phức $z$. Phần thực của số phức $z$ là
\choice
{$2$}
{$-2$}
{$3$}
{\True $-3$}
\loigiai{Noi dung loi giai.
}
\end{ex}
\begin{ex}
 Trên mặt phẳng tọa độ, cho $M(-1,2)$ là điểm biểu diển của số phức $z$. Phần thực của số phức $z$ là
\choice
{$2$}
{$-2$}
{\True $-1$}
{$1$}
\loigiai{Noi dung loi giai.
}
\end{ex}
\begin{ex}
 Trên mặt phẳng tọa độ, cho $M(1,2)$ là điểm biểu diển của số phức $z$. Phần thực của số phức $z$ là
\choice
{$-2$}
{\True $1$}
{$-1$}
{$2$}
\loigiai{Noi dung loi giai.
}
\end{ex}
\begin{ex}
 Trên mặt phẳng tọa độ, cho $M(-3,-1)$ là điểm biểu diển của số phức $z$. Phần thực của số phức $z$ là
\choice
{$3$}
{$1$}
{\True $-3$}
{$-1$}
\loigiai{Noi dung loi giai.
}
\end{ex}
\begin{ex}
 Trên mặt phẳng tọa độ, cho $M(2,-3)$ là điểm biểu diển của số phức $z$. Phần thực của số phức $z$ là
\choice
{$-3$}
{$-2$}
{$3$}
{\True $2$}
\loigiai{Noi dung loi giai.
}
\end{ex}
\begin{ex}
 Trên mặt phẳng tọa độ, cho $M(-2,-1)$ là điểm biểu diển của số phức $z$. Phần thực của số phức $z$ là
\choice
{\True $-2$}
{$1$}
{$2$}
{$-1$}
\loigiai{Noi dung loi giai.
}
\end{ex}
\begin{ex}
 Trên mặt phẳng tọa độ, cho $M(-3,-1)$ là điểm biểu diển của số phức $z$. Phần thực của số phức $z$ là
\choice
{$-1$}
{$3$}
{\True $-3$}
{$1$}
\loigiai{Noi dung loi giai.
}
\end{ex}
\begin{ex}
 Trên mặt phẳng tọa độ, cho $M(2,-3)$ là điểm biểu diển của số phức $z$. Phần thực của số phức $z$ là
\choice
{$-2$}
{$-3$}
{\True $2$}
{$3$}
\loigiai{Noi dung loi giai.
}
\end{ex}
\begin{ex}
 Trên mặt phẳng tọa độ, cho $M(-1,-2)$ là điểm biểu diển của số phức $z$. Phần thực của số phức $z$ là
\choice
{$1$}
{$-2$}
{\True $-1$}
{$2$}
\loigiai{Noi dung loi giai.
}
\end{ex}
\begin{ex}
 Trên mặt phẳng tọa độ, cho $M(-2,-1)$ là điểm biểu diển của số phức $z$. Phần thực của số phức $z$ là
\choice
{$1$}
{$-1$}
{\True $-2$}
{$2$}
\loigiai{Noi dung loi giai.
}
\end{ex}
\begin{ex}
 Trên mặt phẳng tọa độ, cho $M(2,-1)$ là điểm biểu diển của số phức $z$. Phần thực của số phức $z$ là
\choice
{$-2$}
{\True $2$}
{$-1$}
{$1$}
\loigiai{Noi dung loi giai.
}
\end{ex}
\begin{ex}
 Trên mặt phẳng tọa độ, cho $M(-3,1)$ là điểm biểu diển của số phức $z$. Phần thực của số phức $z$ là
\choice
{$1$}
{$3$}
{$-1$}
{\True $-3$}
\loigiai{Noi dung loi giai.
}
\end{ex}
\begin{ex}
 Trên mặt phẳng tọa độ, cho $M(-1,-3)$ là điểm biểu diển của số phức $z$. Phần thực của số phức $z$ là
\choice
{$1$}
{$3$}
{\True $-1$}
{$-3$}
\loigiai{Noi dung loi giai.
}
\end{ex}
\begin{ex}
 Trên mặt phẳng tọa độ, cho $M(2,-1)$ là điểm biểu diển của số phức $z$. Phần thực của số phức $z$ là
\choice
{$1$}
{$-2$}
{\True $2$}
{$-1$}
\loigiai{Noi dung loi giai.
}
\end{ex}
\begin{ex}
 Trên mặt phẳng tọa độ, cho $M(2,-1)$ là điểm biểu diển của số phức $z$. Phần thực của số phức $z$ là
\choice
{\True $2$}
{$1$}
{$-2$}
{$-1$}
\loigiai{Noi dung loi giai.
}
\end{ex}

