\documentclass[12pt,a4paper]{article}
\usepackage[utf8]{vietnam}
\usepackage{amsmath,amssymb,tkz-tab}
\usepackage{fancyhdr}
\usepackage[top=2.5cm, bottom=2.5cm, left=2cm, right=1.5cm] {geometry}
\usepackage[loigiai]{ex_test}
\usetikzlibrary{calc,angles}
\renewcommand{\TrueEX}{\stepcounter{dapan}{\small\circled{\textbf{\Alph{dapan}}}}\ignorespaces}
\renewcommand{\FalseEX}{\stepcounter{dapan}{\small\circled{\textbf{\Alph{dapan}}}}\ignorespaces}
\renewcommand{\footrulewidth}{0.4pt}
\renewtheorem{ex}{\color{blue}Câu}
\begin{document}
\Opensolutionfile{ansbook}[ans/LG]
\Opensolutionfile{ans}[ans/DA]
\pagestyle{fancy}
\fancyhf{}
\fancyfoot[R]{Trang \thepage}
\begin{center}
{\bf\Large BÀI TẬP NGẪU NHIÊN}
\end{center}
\lhead{Biên soạn: TH.s Võ Thanh Phong}
\rhead{THPT Huỳnh THị Hưởng}
\lfoot{Số điện thoại 0981.011.235}
\begin{ex}
Đạo hàm của hàm số $y=\log_9(-3x-1)$ là
\choice
{$y'=\dfrac{3}{3x+1}$}
{\True $y'=\dfrac{3}{(3x+1)\ln 9}$}
{$y'=\dfrac{3\ln{9}}{3x+1}$}
{$y'=\dfrac{3}{\ln{9}}$}
\loigiai{Đạo hàm của hàm số trên là $y'=\dfrac{3}{(3x+1)\ln 9}$.}
\end{ex}
\begin{ex}
Đạo hàm của hàm số $y=\log_{13}(-x)$ là
\choice
{$y'=\dfrac{1}{x}$}
{\True $y'=\dfrac{1}{x\ln 13}$}
{$y'=\dfrac{1}{\ln{13}}$}
{$y'=\dfrac{\ln{13}}{x}$}
\loigiai{Đạo hàm của hàm số trên là $y'=\dfrac{1}{x\ln 13}$.}
\end{ex}
\begin{ex}
Đạo hàm của hàm số $y=\log_{39}(3x+3)$ là
\choice
{$y'=\dfrac{1}{\ln{39}}$}
{$y'=\dfrac{\ln{39}}{x+1}$}
{\True $y'=\dfrac{1}{(x+1)\ln 39}$}
{$y'=\dfrac{1}{x+1}$}
\loigiai{Đạo hàm của hàm số trên là $y'=\dfrac{1}{(x+1)\ln 39}$.}
\end{ex}
\begin{ex}
Đạo hàm của hàm số $y=\log_{27}(-x)$ là
\choice
{$y'=\dfrac{\ln{27}}{x}$}
{$y'=\dfrac{1}{x}$}
{\True $y'=\dfrac{1}{x\ln 27}$}
{$y'=\dfrac{1}{\ln{27}}$}
\loigiai{Đạo hàm của hàm số trên là $y'=\dfrac{1}{x\ln 27}$.}
\end{ex}
\begin{ex}
Đạo hàm của hàm số $y=\log_{49}(4x-4)$ là
\choice
{\True $y'=\dfrac{1}{(x-1)\ln 49}$}
{$y'=\dfrac{1}{x-1}$}
{$y'=\dfrac{1}{\ln{49}}$}
{$y'=\dfrac{\ln{49}}{x-1}$}
\loigiai{Đạo hàm của hàm số trên là $y'=\dfrac{1}{(x-1)\ln 49}$.}
\end{ex}
\begin{ex}
Đạo hàm của hàm số $y=\log_2(x+2)$ là
\choice
{$y'=\dfrac{1}{\ln{2}}$}
{$y'=\dfrac{\ln{2}}{x+2}$}
{\True $y'=\dfrac{1}{(x+2)\ln 2}$}
{$y'=\dfrac{1}{x+2}$}
\loigiai{Đạo hàm của hàm số trên là $y'=\dfrac{1}{(x+2)\ln 2}$.}
\end{ex}
\begin{ex}
Đạo hàm của hàm số $y=\log_7(x-2)$ là
\choice
{$y'=\dfrac{1}{\ln{7}}$}
{\True $y'=\dfrac{1}{(x-2)\ln 7}$}
{$y'=\dfrac{1}{x-2}$}
{$y'=\dfrac{\ln{7}}{x-2}$}
\loigiai{Đạo hàm của hàm số trên là $y'=\dfrac{1}{(x-2)\ln 7}$.}
\end{ex}
\begin{ex}
Đạo hàm của hàm số $y=\log_{39}(-4x+3)$ là
\choice
{$y'=\dfrac{4}{4x-3}$}
{\True $y'=\dfrac{4}{(4x-3)\ln 39}$}
{$y'=\dfrac{4\ln{39}}{4x-3}$}
{$y'=\dfrac{4}{\ln{39}}$}
\loigiai{Đạo hàm của hàm số trên là $y'=\dfrac{4}{(4x-3)\ln 39}$.}
\end{ex}
\begin{ex}
Đạo hàm của hàm số $y=\log_{27}(-5x-3)$ là
\choice
{$y'=\dfrac{5\ln{27}}{5x+3}$}
{$y'=\dfrac{5}{\ln{27}}$}
{$y'=\dfrac{5}{5x+3}$}
{\True $y'=\dfrac{5}{(5x+3)\ln 27}$}
\loigiai{Đạo hàm của hàm số trên là $y'=\dfrac{5}{(5x+3)\ln 27}$.}
\end{ex}
\begin{ex}
Đạo hàm của hàm số $y=\log_{40}(-2x-4)$ là
\choice
{\True $y'=\dfrac{1}{(x+2)\ln 40}$}
{$y'=\dfrac{\ln{40}}{x+2}$}
{$y'=\dfrac{1}{\ln{40}}$}
{$y'=\dfrac{1}{x+2}$}
\loigiai{Đạo hàm của hàm số trên là $y'=\dfrac{1}{(x+2)\ln 40}$.}
\end{ex}
\begin{ex}
Đạo hàm của hàm số $y=\log_{18}(-x+3)$ là
\choice
{$y'=\dfrac{1}{\ln{18}}$}
{$y'=\dfrac{\ln{18}}{x-3}$}
{\True $y'=\dfrac{1}{(x-3)\ln 18}$}
{$y'=\dfrac{1}{x-3}$}
\loigiai{Đạo hàm của hàm số trên là $y'=\dfrac{1}{(x-3)\ln 18}$.}
\end{ex}
\begin{ex}
Đạo hàm của hàm số $y=\log_{49}(x-5)$ là
\choice
{\True $y'=\dfrac{1}{(x-5)\ln 49}$}
{$y'=\dfrac{1}{x-5}$}
{$y'=\dfrac{\ln{49}}{x-5}$}
{$y'=\dfrac{1}{\ln{49}}$}
\loigiai{Đạo hàm của hàm số trên là $y'=\dfrac{1}{(x-5)\ln 49}$.}
\end{ex}
\begin{ex}
Đạo hàm của hàm số $y=\log_3(2x-3)$ là
\choice
{\True $y'=\dfrac{2}{(2x-3)\ln 3}$}
{$y'=\dfrac{2}{2x-3}$}
{$y'=\dfrac{2}{\ln{3}}$}
{$y'=\dfrac{2\ln{3}}{2x-3}$}
\loigiai{Đạo hàm của hàm số trên là $y'=\dfrac{2}{(2x-3)\ln 3}$.}
\end{ex}
\begin{ex}
Đạo hàm của hàm số $y=\log_7(5x-3)$ là
\choice
{$y'=\dfrac{5}{5x-3}$}
{\True $y'=\dfrac{5}{(5x-3)\ln 7}$}
{$y'=\dfrac{5\ln{7}}{5x-3}$}
{$y'=\dfrac{5}{\ln{7}}$}
\loigiai{Đạo hàm của hàm số trên là $y'=\dfrac{5}{(5x-3)\ln 7}$.}
\end{ex}
\begin{ex}
Đạo hàm của hàm số $y=\log_{13}(-5x-5)$ là
\choice
{$y'=\dfrac{1}{\ln{13}}$}
{\True $y'=\dfrac{1}{(x+1)\ln 13}$}
{$y'=\dfrac{\ln{13}}{x+1}$}
{$y'=\dfrac{1}{x+1}$}
\loigiai{Đạo hàm của hàm số trên là $y'=\dfrac{1}{(x+1)\ln 13}$.}
\end{ex}
\begin{ex}
Đạo hàm của hàm số $y=\log_{37}(-x-5)$ là
\choice
{$y'=\dfrac{1}{x+5}$}
{$y'=\dfrac{1}{\ln{37}}$}
{\True $y'=\dfrac{1}{(x+5)\ln 37}$}
{$y'=\dfrac{\ln{37}}{x+5}$}
\loigiai{Đạo hàm của hàm số trên là $y'=\dfrac{1}{(x+5)\ln 37}$.}
\end{ex}
\begin{ex}
Đạo hàm của hàm số $y=\log_6(-4x+1)$ là
\choice
{$y'=\dfrac{4}{\ln{6}}$}
{\True $y'=\dfrac{4}{(4x-1)\ln 6}$}
{$y'=\dfrac{4\ln{6}}{4x-1}$}
{$y'=\dfrac{4}{4x-1}$}
\loigiai{Đạo hàm của hàm số trên là $y'=\dfrac{4}{(4x-1)\ln 6}$.}
\end{ex}
\begin{ex}
Đạo hàm của hàm số $y=\log_4(-2x+5)$ là
\choice
{$y'=\dfrac{2}{2x-5}$}
{$y'=\dfrac{2}{\ln{4}}$}
{\True $y'=\dfrac{2}{(2x-5)\ln 4}$}
{$y'=\dfrac{2\ln{4}}{2x-5}$}
\loigiai{Đạo hàm của hàm số trên là $y'=\dfrac{2}{(2x-5)\ln 4}$.}
\end{ex}
\begin{ex}
Đạo hàm của hàm số $y=\log_{40}(-5x-2)$ là
\choice
{$y'=\dfrac{5\ln{40}}{5x+2}$}
{$y'=\dfrac{5}{5x+2}$}
{\True $y'=\dfrac{5}{(5x+2)\ln 40}$}
{$y'=\dfrac{5}{\ln{40}}$}
\loigiai{Đạo hàm của hàm số trên là $y'=\dfrac{5}{(5x+2)\ln 40}$.}
\end{ex}
\begin{ex}
Đạo hàm của hàm số $y=\log_6(-4x)$ là
\choice
{\True $y'=\dfrac{1}{x\ln 6}$}
{$y'=\dfrac{1}{x}$}
{$y'=\dfrac{\ln{6}}{x}$}
{$y'=\dfrac{1}{\ln{6}}$}
\loigiai{Đạo hàm của hàm số trên là $y'=\dfrac{1}{x\ln 6}$.}
\end{ex}
\begin{ex}
Đạo hàm của hàm số $y=\log_{34}(3x+5)$ là
\choice
{$y'=\dfrac{3}{\ln{34}}$}
{$y'=\dfrac{3}{3x+5}$}
{$y'=\dfrac{3\ln{34}}{3x+5}$}
{\True $y'=\dfrac{3}{(3x+5)\ln 34}$}
\loigiai{Đạo hàm của hàm số trên là $y'=\dfrac{3}{(3x+5)\ln 34}$.}
\end{ex}
\begin{ex}
Đạo hàm của hàm số $y=\log_{17}(-2x+2)$ là
\choice
{$y'=\dfrac{1}{\ln{17}}$}
{$y'=\dfrac{\ln{17}}{x-1}$}
{\True $y'=\dfrac{1}{(x-1)\ln 17}$}
{$y'=\dfrac{1}{x-1}$}
\loigiai{Đạo hàm của hàm số trên là $y'=\dfrac{1}{(x-1)\ln 17}$.}
\end{ex}
\begin{ex}
Đạo hàm của hàm số $y=\log_{36}(-5x-4)$ là
\choice
{$y'=\dfrac{5\ln{36}}{5x+4}$}
{\True $y'=\dfrac{5}{(5x+4)\ln 36}$}
{$y'=\dfrac{5}{5x+4}$}
{$y'=\dfrac{5}{\ln{36}}$}
\loigiai{Đạo hàm của hàm số trên là $y'=\dfrac{5}{(5x+4)\ln 36}$.}
\end{ex}
\begin{ex}
Đạo hàm của hàm số $y=\log_{35}(-5x+4)$ là
\choice
{$y'=\dfrac{5\ln{35}}{5x-4}$}
{$y'=\dfrac{5}{\ln{35}}$}
{\True $y'=\dfrac{5}{(5x-4)\ln 35}$}
{$y'=\dfrac{5}{5x-4}$}
\loigiai{Đạo hàm của hàm số trên là $y'=\dfrac{5}{(5x-4)\ln 35}$.}
\end{ex}
\begin{ex}
Đạo hàm của hàm số $y=\log_{27}(4x-3)$ là
\choice
{$y'=\dfrac{4\ln{27}}{4x-3}$}
{$y'=\dfrac{4}{\ln{27}}$}
{\True $y'=\dfrac{4}{(4x-3)\ln 27}$}
{$y'=\dfrac{4}{4x-3}$}
\loigiai{Đạo hàm của hàm số trên là $y'=\dfrac{4}{(4x-3)\ln 27}$.}
\end{ex}
\begin{ex}
Đạo hàm của hàm số $y=\log_{43}(-2x+2)$ là
\choice
{$y'=\dfrac{1}{x-1}$}
{$y'=\dfrac{1}{\ln{43}}$}
{\True $y'=\dfrac{1}{(x-1)\ln 43}$}
{$y'=\dfrac{\ln{43}}{x-1}$}
\loigiai{Đạo hàm của hàm số trên là $y'=\dfrac{1}{(x-1)\ln 43}$.}
\end{ex}
\begin{ex}
Đạo hàm của hàm số $y=\log_2(-4x-3)$ là
\choice
{$y'=\dfrac{4\ln{2}}{4x+3}$}
{\True $y'=\dfrac{4}{(4x+3)\ln 2}$}
{$y'=\dfrac{4}{\ln{2}}$}
{$y'=\dfrac{4}{4x+3}$}
\loigiai{Đạo hàm của hàm số trên là $y'=\dfrac{4}{(4x+3)\ln 2}$.}
\end{ex}
\begin{ex}
Đạo hàm của hàm số $y=\log_8(-2x+1)$ là
\choice
{$y'=\dfrac{2}{2x-1}$}
{\True $y'=\dfrac{2}{(2x-1)\ln 8}$}
{$y'=\dfrac{2\ln{8}}{2x-1}$}
{$y'=\dfrac{2}{\ln{8}}$}
\loigiai{Đạo hàm của hàm số trên là $y'=\dfrac{2}{(2x-1)\ln 8}$.}
\end{ex}
\begin{ex}
Đạo hàm của hàm số $y=\log_{40}(-x-4)$ là
\choice
{$y'=\dfrac{1}{\ln{40}}$}
{$y'=\dfrac{\ln{40}}{x+4}$}
{$y'=\dfrac{1}{x+4}$}
{\True $y'=\dfrac{1}{(x+4)\ln 40}$}
\loigiai{Đạo hàm của hàm số trên là $y'=\dfrac{1}{(x+4)\ln 40}$.}
\end{ex}
\begin{ex}
Đạo hàm của hàm số $y=\log_{34}(-3x-2)$ là
\choice
{$y'=\dfrac{3\ln{34}}{3x+2}$}
{$y'=\dfrac{3}{\ln{34}}$}
{\True $y'=\dfrac{3}{(3x+2)\ln 34}$}
{$y'=\dfrac{3}{3x+2}$}
\loigiai{Đạo hàm của hàm số trên là $y'=\dfrac{3}{(3x+2)\ln 34}$.}
\end{ex}
\begin{ex}
Đạo hàm của hàm số $y=\log_{12}(4x+2)$ là
\choice
{$y'=\dfrac{2}{2x+1}$}
{$y'=\dfrac{2}{\ln{12}}$}
{$y'=\dfrac{2\ln{12}}{2x+1}$}
{\True $y'=\dfrac{2}{(2x+1)\ln 12}$}
\loigiai{Đạo hàm của hàm số trên là $y'=\dfrac{2}{(2x+1)\ln 12}$.}
\end{ex}
\begin{ex}
Đạo hàm của hàm số $y=\log_{22}(-3x+2)$ là
\choice
{$y'=\dfrac{3}{3x-2}$}
{\True $y'=\dfrac{3}{(3x-2)\ln 22}$}
{$y'=\dfrac{3}{\ln{22}}$}
{$y'=\dfrac{3\ln{22}}{3x-2}$}
\loigiai{Đạo hàm của hàm số trên là $y'=\dfrac{3}{(3x-2)\ln 22}$.}
\end{ex}
\begin{ex}
Đạo hàm của hàm số $y=\log_{48}(-4x+1)$ là
\choice
{\True $y'=\dfrac{4}{(4x-1)\ln 48}$}
{$y'=\dfrac{4}{\ln{48}}$}
{$y'=\dfrac{4\ln{48}}{4x-1}$}
{$y'=\dfrac{4}{4x-1}$}
\loigiai{Đạo hàm của hàm số trên là $y'=\dfrac{4}{(4x-1)\ln 48}$.}
\end{ex}
\begin{ex}
Đạo hàm của hàm số $y=\log_4(-2x+2)$ là
\choice
{$y'=\dfrac{1}{\ln{4}}$}
{$y'=\dfrac{1}{x-1}$}
{\True $y'=\dfrac{1}{(x-1)\ln 4}$}
{$y'=\dfrac{\ln{4}}{x-1}$}
\loigiai{Đạo hàm của hàm số trên là $y'=\dfrac{1}{(x-1)\ln 4}$.}
\end{ex}
\begin{ex}
Đạo hàm của hàm số $y=\log_{32}(4x+3)$ là
\choice
{\True $y'=\dfrac{4}{(4x+3)\ln 32}$}
{$y'=\dfrac{4}{4x+3}$}
{$y'=\dfrac{4\ln{32}}{4x+3}$}
{$y'=\dfrac{4}{\ln{32}}$}
\loigiai{Đạo hàm của hàm số trên là $y'=\dfrac{4}{(4x+3)\ln 32}$.}
\end{ex}
\begin{ex}
Đạo hàm của hàm số $y=\log_{17}(4x+2)$ là
\choice
{\True $y'=\dfrac{2}{(2x+1)\ln 17}$}
{$y'=\dfrac{2\ln{17}}{2x+1}$}
{$y'=\dfrac{2}{\ln{17}}$}
{$y'=\dfrac{2}{2x+1}$}
\loigiai{Đạo hàm của hàm số trên là $y'=\dfrac{2}{(2x+1)\ln 17}$.}
\end{ex}
\begin{ex}
Đạo hàm của hàm số $y=\log_{36}(2x-1)$ là
\choice
{$y'=\dfrac{2}{2x-1}$}
{\True $y'=\dfrac{2}{(2x-1)\ln 36}$}
{$y'=\dfrac{2}{\ln{36}}$}
{$y'=\dfrac{2\ln{36}}{2x-1}$}
\loigiai{Đạo hàm của hàm số trên là $y'=\dfrac{2}{(2x-1)\ln 36}$.}
\end{ex}
\begin{ex}
Đạo hàm của hàm số $y=\log_9(-3x+3)$ là
\choice
{$y'=\dfrac{\ln{9}}{x-1}$}
{$y'=\dfrac{1}{x-1}$}
{\True $y'=\dfrac{1}{(x-1)\ln 9}$}
{$y'=\dfrac{1}{\ln{9}}$}
\loigiai{Đạo hàm của hàm số trên là $y'=\dfrac{1}{(x-1)\ln 9}$.}
\end{ex}
\begin{ex}
Đạo hàm của hàm số $y=\log_{42}(3x-2)$ là
\choice
{$y'=\dfrac{3\ln{42}}{3x-2}$}
{\True $y'=\dfrac{3}{(3x-2)\ln 42}$}
{$y'=\dfrac{3}{3x-2}$}
{$y'=\dfrac{3}{\ln{42}}$}
\loigiai{Đạo hàm của hàm số trên là $y'=\dfrac{3}{(3x-2)\ln 42}$.}
\end{ex}
\begin{ex}
Đạo hàm của hàm số $y=\log_{30}(-3x-2)$ là
\choice
{$y'=\dfrac{3}{3x+2}$}
{$y'=\dfrac{3\ln{30}}{3x+2}$}
{$y'=\dfrac{3}{\ln{30}}$}
{\True $y'=\dfrac{3}{(3x+2)\ln 30}$}
\loigiai{Đạo hàm của hàm số trên là $y'=\dfrac{3}{(3x+2)\ln 30}$.}
\end{ex}
\begin{ex}
Đạo hàm của hàm số $y=\log_{43}(-4x-2)$ là
\choice
{\True $y'=\dfrac{2}{(2x+1)\ln 43}$}
{$y'=\dfrac{2}{\ln{43}}$}
{$y'=\dfrac{2}{2x+1}$}
{$y'=\dfrac{2\ln{43}}{2x+1}$}
\loigiai{Đạo hàm của hàm số trên là $y'=\dfrac{2}{(2x+1)\ln 43}$.}
\end{ex}
\begin{ex}
Đạo hàm của hàm số $y=\log_{21}(-x-4)$ là
\choice
{$y'=\dfrac{\ln{21}}{x+4}$}
{$y'=\dfrac{1}{\ln{21}}$}
{\True $y'=\dfrac{1}{(x+4)\ln 21}$}
{$y'=\dfrac{1}{x+4}$}
\loigiai{Đạo hàm của hàm số trên là $y'=\dfrac{1}{(x+4)\ln 21}$.}
\end{ex}
\begin{ex}
Đạo hàm của hàm số $y=\log_{47}(x-1)$ là
\choice
{$y'=\dfrac{1}{x-1}$}
{\True $y'=\dfrac{1}{(x-1)\ln 47}$}
{$y'=\dfrac{\ln{47}}{x-1}$}
{$y'=\dfrac{1}{\ln{47}}$}
\loigiai{Đạo hàm của hàm số trên là $y'=\dfrac{1}{(x-1)\ln 47}$.}
\end{ex}
\begin{ex}
Đạo hàm của hàm số $y=\log_{22}(-4x-1)$ là
\choice
{$y'=\dfrac{4}{\ln{22}}$}
{$y'=\dfrac{4}{4x+1}$}
{\True $y'=\dfrac{4}{(4x+1)\ln 22}$}
{$y'=\dfrac{4\ln{22}}{4x+1}$}
\loigiai{Đạo hàm của hàm số trên là $y'=\dfrac{4}{(4x+1)\ln 22}$.}
\end{ex}
\begin{ex}
Đạo hàm của hàm số $y=\log_{35}(-3x-1)$ là
\choice
{$y'=\dfrac{3}{3x+1}$}
{$y'=\dfrac{3}{\ln{35}}$}
{$y'=\dfrac{3\ln{35}}{3x+1}$}
{\True $y'=\dfrac{3}{(3x+1)\ln 35}$}
\loigiai{Đạo hàm của hàm số trên là $y'=\dfrac{3}{(3x+1)\ln 35}$.}
\end{ex}
\begin{ex}
Đạo hàm của hàm số $y=\log_{14}(-3x-3)$ là
\choice
{\True $y'=\dfrac{1}{(x+1)\ln 14}$}
{$y'=\dfrac{1}{\ln{14}}$}
{$y'=\dfrac{1}{x+1}$}
{$y'=\dfrac{\ln{14}}{x+1}$}
\loigiai{Đạo hàm của hàm số trên là $y'=\dfrac{1}{(x+1)\ln 14}$.}
\end{ex}
\begin{ex}
Đạo hàm của hàm số $y=\log_{29}(4x+2)$ là
\choice
{\True $y'=\dfrac{2}{(2x+1)\ln 29}$}
{$y'=\dfrac{2}{2x+1}$}
{$y'=\dfrac{2}{\ln{29}}$}
{$y'=\dfrac{2\ln{29}}{2x+1}$}
\loigiai{Đạo hàm của hàm số trên là $y'=\dfrac{2}{(2x+1)\ln 29}$.}
\end{ex}
\begin{ex}
Đạo hàm của hàm số $y=\log_{22}(-3x+4)$ là
\choice
{\True $y'=\dfrac{3}{(3x-4)\ln 22}$}
{$y'=\dfrac{3\ln{22}}{3x-4}$}
{$y'=\dfrac{3}{\ln{22}}$}
{$y'=\dfrac{3}{3x-4}$}
\loigiai{Đạo hàm của hàm số trên là $y'=\dfrac{3}{(3x-4)\ln 22}$.}
\end{ex}
\begin{ex}
Đạo hàm của hàm số $y=\log_{25}(2x-5)$ là
\choice
{\True $y'=\dfrac{2}{(2x-5)\ln 25}$}
{$y'=\dfrac{2\ln{25}}{2x-5}$}
{$y'=\dfrac{2}{\ln{25}}$}
{$y'=\dfrac{2}{2x-5}$}
\loigiai{Đạo hàm của hàm số trên là $y'=\dfrac{2}{(2x-5)\ln 25}$.}
\end{ex}
\begin{ex}
Đạo hàm của hàm số $y=\log_2(x-4)$ là
\choice
{$y'=\dfrac{1}{x-4}$}
{\True $y'=\dfrac{1}{(x-4)\ln 2}$}
{$y'=\dfrac{\ln{2}}{x-4}$}
{$y'=\dfrac{1}{\ln{2}}$}
\loigiai{Đạo hàm của hàm số trên là $y'=\dfrac{1}{(x-4)\ln 2}$.}
\end{ex}
\begin{ex}
Đạo hàm của hàm số $y=\log_{27}(-2x-4)$ là
\choice
{$y'=\dfrac{1}{\ln{27}}$}
{$y'=\dfrac{\ln{27}}{x+2}$}
{\True $y'=\dfrac{1}{(x+2)\ln 27}$}
{$y'=\dfrac{1}{x+2}$}
\loigiai{Đạo hàm của hàm số trên là $y'=\dfrac{1}{(x+2)\ln 27}$.}
\end{ex}
\begin{ex}
Đạo hàm của hàm số $y=\log_{24}(2x+1)$ là
\choice
{$y'=\dfrac{2\ln{24}}{2x+1}$}
{$y'=\dfrac{2}{\ln{24}}$}
{\True $y'=\dfrac{2}{(2x+1)\ln 24}$}
{$y'=\dfrac{2}{2x+1}$}
\loigiai{Đạo hàm của hàm số trên là $y'=\dfrac{2}{(2x+1)\ln 24}$.}
\end{ex}
\begin{ex}
Đạo hàm của hàm số $y=\log_{23}(5x+2)$ là
\choice
{\True $y'=\dfrac{5}{(5x+2)\ln 23}$}
{$y'=\dfrac{5}{5x+2}$}
{$y'=\dfrac{5}{\ln{23}}$}
{$y'=\dfrac{5\ln{23}}{5x+2}$}
\loigiai{Đạo hàm của hàm số trên là $y'=\dfrac{5}{(5x+2)\ln 23}$.}
\end{ex}
\begin{ex}
Đạo hàm của hàm số $y=\log_{26}(-x-2)$ là
\choice
{\True $y'=\dfrac{1}{(x+2)\ln 26}$}
{$y'=\dfrac{1}{x+2}$}
{$y'=\dfrac{1}{\ln{26}}$}
{$y'=\dfrac{\ln{26}}{x+2}$}
\loigiai{Đạo hàm của hàm số trên là $y'=\dfrac{1}{(x+2)\ln 26}$.}
\end{ex}
\begin{ex}
Đạo hàm của hàm số $y=\log_9(3x-2)$ là
\choice
{$y'=\dfrac{3}{3x-2}$}
{$y'=\dfrac{3\ln{9}}{3x-2}$}
{$y'=\dfrac{3}{\ln{9}}$}
{\True $y'=\dfrac{3}{(3x-2)\ln 9}$}
\loigiai{Đạo hàm của hàm số trên là $y'=\dfrac{3}{(3x-2)\ln 9}$.}
\end{ex}
\begin{ex}
Đạo hàm của hàm số $y=\log_{12}(4x+5)$ là
\choice
{$y'=\dfrac{4\ln{12}}{4x+5}$}
{$y'=\dfrac{4}{4x+5}$}
{$y'=\dfrac{4}{\ln{12}}$}
{\True $y'=\dfrac{4}{(4x+5)\ln 12}$}
\loigiai{Đạo hàm của hàm số trên là $y'=\dfrac{4}{(4x+5)\ln 12}$.}
\end{ex}
\begin{ex}
Đạo hàm của hàm số $y=\log_{20}(-4x-2)$ là
\choice
{$y'=\dfrac{2}{2x+1}$}
{$y'=\dfrac{2\ln{20}}{2x+1}$}
{\True $y'=\dfrac{2}{(2x+1)\ln 20}$}
{$y'=\dfrac{2}{\ln{20}}$}
\loigiai{Đạo hàm của hàm số trên là $y'=\dfrac{2}{(2x+1)\ln 20}$.}
\end{ex}
\begin{ex}
Đạo hàm của hàm số $y=\log_{31}(-x-2)$ là
\choice
{$y'=\dfrac{1}{\ln{31}}$}
{$y'=\dfrac{1}{x+2}$}
{\True $y'=\dfrac{1}{(x+2)\ln 31}$}
{$y'=\dfrac{\ln{31}}{x+2}$}
\loigiai{Đạo hàm của hàm số trên là $y'=\dfrac{1}{(x+2)\ln 31}$.}
\end{ex}
\begin{ex}
Đạo hàm của hàm số $y=\log_{16}(3x+2)$ là
\choice
{\True $y'=\dfrac{3}{(3x+2)\ln 16}$}
{$y'=\dfrac{3\ln{16}}{3x+2}$}
{$y'=\dfrac{3}{\ln{16}}$}
{$y'=\dfrac{3}{3x+2}$}
\loigiai{Đạo hàm của hàm số trên là $y'=\dfrac{3}{(3x+2)\ln 16}$.}
\end{ex}
\begin{ex}
Đạo hàm của hàm số $y=\log_{30}(4x+3)$ là
\choice
{$y'=\dfrac{4\ln{30}}{4x+3}$}
{\True $y'=\dfrac{4}{(4x+3)\ln 30}$}
{$y'=\dfrac{4}{\ln{30}}$}
{$y'=\dfrac{4}{4x+3}$}
\loigiai{Đạo hàm của hàm số trên là $y'=\dfrac{4}{(4x+3)\ln 30}$.}
\end{ex}
\begin{ex}
Đạo hàm của hàm số $y=\log_{13}(-5x+5)$ là
\choice
{$y'=\dfrac{1}{x-1}$}
{$y'=\dfrac{1}{\ln{13}}$}
{\True $y'=\dfrac{1}{(x-1)\ln 13}$}
{$y'=\dfrac{\ln{13}}{x-1}$}
\loigiai{Đạo hàm của hàm số trên là $y'=\dfrac{1}{(x-1)\ln 13}$.}
\end{ex}
\begin{ex}
Đạo hàm của hàm số $y=\log_{43}(3x-3)$ là
\choice
{$y'=\dfrac{1}{x-1}$}
{\True $y'=\dfrac{1}{(x-1)\ln 43}$}
{$y'=\dfrac{1}{\ln{43}}$}
{$y'=\dfrac{\ln{43}}{x-1}$}
\loigiai{Đạo hàm của hàm số trên là $y'=\dfrac{1}{(x-1)\ln 43}$.}
\end{ex}
\begin{ex}
Đạo hàm của hàm số $y=\log_{27}(x-3)$ là
\choice
{$y'=\dfrac{\ln{27}}{x-3}$}
{$y'=\dfrac{1}{\ln{27}}$}
{$y'=\dfrac{1}{x-3}$}
{\True $y'=\dfrac{1}{(x-3)\ln 27}$}
\loigiai{Đạo hàm của hàm số trên là $y'=\dfrac{1}{(x-3)\ln 27}$.}
\end{ex}
\begin{ex}
Đạo hàm của hàm số $y=\log_{20}(-3x-1)$ là
\choice
{\True $y'=\dfrac{3}{(3x+1)\ln 20}$}
{$y'=\dfrac{3}{3x+1}$}
{$y'=\dfrac{3}{\ln{20}}$}
{$y'=\dfrac{3\ln{20}}{3x+1}$}
\loigiai{Đạo hàm của hàm số trên là $y'=\dfrac{3}{(3x+1)\ln 20}$.}
\end{ex}
\begin{ex}
Đạo hàm của hàm số $y=\log_8(3x+1)$ là
\choice
{$y'=\dfrac{3\ln{8}}{3x+1}$}
{$y'=\dfrac{3}{3x+1}$}
{$y'=\dfrac{3}{\ln{8}}$}
{\True $y'=\dfrac{3}{(3x+1)\ln 8}$}
\loigiai{Đạo hàm của hàm số trên là $y'=\dfrac{3}{(3x+1)\ln 8}$.}
\end{ex}
\begin{ex}
Đạo hàm của hàm số $y=\log_{23}(-2x+4)$ là
\choice
{$y'=\dfrac{1}{x-2}$}
{\True $y'=\dfrac{1}{(x-2)\ln 23}$}
{$y'=\dfrac{1}{\ln{23}}$}
{$y'=\dfrac{\ln{23}}{x-2}$}
\loigiai{Đạo hàm của hàm số trên là $y'=\dfrac{1}{(x-2)\ln 23}$.}
\end{ex}
\begin{ex}
Đạo hàm của hàm số $y=\log_{40}x$ là
\choice
{\True $y'=\dfrac{1}{x\ln 40}$}
{$y'=\dfrac{\ln{40}}{x}$}
{$y'=\dfrac{1}{\ln{40}}$}
{$y'=\dfrac{1}{x}$}
\loigiai{Đạo hàm của hàm số trên là $y'=\dfrac{1}{x\ln 40}$.}
\end{ex}
\begin{ex}
Đạo hàm của hàm số $y=\log_{44}(2x+1)$ là
\choice
{$y'=\dfrac{2}{\ln{44}}$}
{\True $y'=\dfrac{2}{(2x+1)\ln 44}$}
{$y'=\dfrac{2}{2x+1}$}
{$y'=\dfrac{2\ln{44}}{2x+1}$}
\loigiai{Đạo hàm của hàm số trên là $y'=\dfrac{2}{(2x+1)\ln 44}$.}
\end{ex}
\begin{ex}
Đạo hàm của hàm số $y=\log_9(-4x-5)$ là
\choice
{$y'=\dfrac{4}{4x+5}$}
{\True $y'=\dfrac{4}{(4x+5)\ln 9}$}
{$y'=\dfrac{4\ln{9}}{4x+5}$}
{$y'=\dfrac{4}{\ln{9}}$}
\loigiai{Đạo hàm của hàm số trên là $y'=\dfrac{4}{(4x+5)\ln 9}$.}
\end{ex}
\begin{ex}
Đạo hàm của hàm số $y=\log_{42}(x+1)$ là
\choice
{\True $y'=\dfrac{1}{(x+1)\ln 42}$}
{$y'=\dfrac{1}{x+1}$}
{$y'=\dfrac{1}{\ln{42}}$}
{$y'=\dfrac{\ln{42}}{x+1}$}
\loigiai{Đạo hàm của hàm số trên là $y'=\dfrac{1}{(x+1)\ln 42}$.}
\end{ex}
\begin{ex}
Đạo hàm của hàm số $y=\log_{39}(-4x-3)$ là
\choice
{$y'=\dfrac{4\ln{39}}{4x+3}$}
{$y'=\dfrac{4}{4x+3}$}
{$y'=\dfrac{4}{\ln{39}}$}
{\True $y'=\dfrac{4}{(4x+3)\ln 39}$}
\loigiai{Đạo hàm của hàm số trên là $y'=\dfrac{4}{(4x+3)\ln 39}$.}
\end{ex}
\begin{ex}
Đạo hàm của hàm số $y=\log_{12}(2x-1)$ là
\choice
{$y'=\dfrac{2\ln{12}}{2x-1}$}
{\True $y'=\dfrac{2}{(2x-1)\ln 12}$}
{$y'=\dfrac{2}{2x-1}$}
{$y'=\dfrac{2}{\ln{12}}$}
\loigiai{Đạo hàm của hàm số trên là $y'=\dfrac{2}{(2x-1)\ln 12}$.}
\end{ex}
\begin{ex}
Đạo hàm của hàm số $y=\log_{45}(-x-2)$ là
\choice
{$y'=\dfrac{\ln{45}}{x+2}$}
{$y'=\dfrac{1}{\ln{45}}$}
{\True $y'=\dfrac{1}{(x+2)\ln 45}$}
{$y'=\dfrac{1}{x+2}$}
\loigiai{Đạo hàm của hàm số trên là $y'=\dfrac{1}{(x+2)\ln 45}$.}
\end{ex}
\begin{ex}
Đạo hàm của hàm số $y=\log_{36}(3x+4)$ là
\choice
{$y'=\dfrac{3\ln{36}}{3x+4}$}
{$y'=\dfrac{3}{\ln{36}}$}
{$y'=\dfrac{3}{3x+4}$}
{\True $y'=\dfrac{3}{(3x+4)\ln 36}$}
\loigiai{Đạo hàm của hàm số trên là $y'=\dfrac{3}{(3x+4)\ln 36}$.}
\end{ex}
\begin{ex}
Đạo hàm của hàm số $y=\log_{27}(-5x)$ là
\choice
{$y'=\dfrac{\ln{27}}{x}$}
{$y'=\dfrac{1}{\ln{27}}$}
{$y'=\dfrac{1}{x}$}
{\True $y'=\dfrac{1}{x\ln 27}$}
\loigiai{Đạo hàm của hàm số trên là $y'=\dfrac{1}{x\ln 27}$.}
\end{ex}
\begin{ex}
Đạo hàm của hàm số $y=\log_6(-4x+2)$ là
\choice
{$y'=\dfrac{2}{\ln{6}}$}
{$y'=\dfrac{2}{2x-1}$}
{\True $y'=\dfrac{2}{(2x-1)\ln 6}$}
{$y'=\dfrac{2\ln{6}}{2x-1}$}
\loigiai{Đạo hàm của hàm số trên là $y'=\dfrac{2}{(2x-1)\ln 6}$.}
\end{ex}
\begin{ex}
Đạo hàm của hàm số $y=\log_{33}(-5x+2)$ là
\choice
{$y'=\dfrac{5}{\ln{33}}$}
{$y'=\dfrac{5\ln{33}}{5x-2}$}
{\True $y'=\dfrac{5}{(5x-2)\ln 33}$}
{$y'=\dfrac{5}{5x-2}$}
\loigiai{Đạo hàm của hàm số trên là $y'=\dfrac{5}{(5x-2)\ln 33}$.}
\end{ex}
\begin{ex}
Đạo hàm của hàm số $y=\log_{45}(-2x+4)$ là
\choice
{\True $y'=\dfrac{1}{(x-2)\ln 45}$}
{$y'=\dfrac{1}{x-2}$}
{$y'=\dfrac{1}{\ln{45}}$}
{$y'=\dfrac{\ln{45}}{x-2}$}
\loigiai{Đạo hàm của hàm số trên là $y'=\dfrac{1}{(x-2)\ln 45}$.}
\end{ex}
\begin{ex}
Đạo hàm của hàm số $y=\log_7(3x-4)$ là
\choice
{\True $y'=\dfrac{3}{(3x-4)\ln 7}$}
{$y'=\dfrac{3}{\ln{7}}$}
{$y'=\dfrac{3\ln{7}}{3x-4}$}
{$y'=\dfrac{3}{3x-4}$}
\loigiai{Đạo hàm của hàm số trên là $y'=\dfrac{3}{(3x-4)\ln 7}$.}
\end{ex}
\begin{ex}
Đạo hàm của hàm số $y=\log_{35}(3x+5)$ là
\choice
{$y'=\dfrac{3}{\ln{35}}$}
{$y'=\dfrac{3}{3x+5}$}
{\True $y'=\dfrac{3}{(3x+5)\ln 35}$}
{$y'=\dfrac{3\ln{35}}{3x+5}$}
\loigiai{Đạo hàm của hàm số trên là $y'=\dfrac{3}{(3x+5)\ln 35}$.}
\end{ex}
\begin{ex}
Đạo hàm của hàm số $y=\log_{33}(4x-1)$ là
\choice
{\True $y'=\dfrac{4}{(4x-1)\ln 33}$}
{$y'=\dfrac{4}{\ln{33}}$}
{$y'=\dfrac{4}{4x-1}$}
{$y'=\dfrac{4\ln{33}}{4x-1}$}
\loigiai{Đạo hàm của hàm số trên là $y'=\dfrac{4}{(4x-1)\ln 33}$.}
\end{ex}
\begin{ex}
Đạo hàm của hàm số $y=\log_7(5x+2)$ là
\choice
{$y'=\dfrac{5}{\ln{7}}$}
{$y'=\dfrac{5}{5x+2}$}
{\True $y'=\dfrac{5}{(5x+2)\ln 7}$}
{$y'=\dfrac{5\ln{7}}{5x+2}$}
\loigiai{Đạo hàm của hàm số trên là $y'=\dfrac{5}{(5x+2)\ln 7}$.}
\end{ex}
\begin{ex}
Đạo hàm của hàm số $y=\log_{19}(2x+2)$ là
\choice
{\True $y'=\dfrac{1}{(x+1)\ln 19}$}
{$y'=\dfrac{\ln{19}}{x+1}$}
{$y'=\dfrac{1}{x+1}$}
{$y'=\dfrac{1}{\ln{19}}$}
\loigiai{Đạo hàm của hàm số trên là $y'=\dfrac{1}{(x+1)\ln 19}$.}
\end{ex}
\begin{ex}
Đạo hàm của hàm số $y=\log_{15}(4x-2)$ là
\choice
{$y'=\dfrac{2\ln{15}}{2x-1}$}
{$y'=\dfrac{2}{2x-1}$}
{$y'=\dfrac{2}{\ln{15}}$}
{\True $y'=\dfrac{2}{(2x-1)\ln 15}$}
\loigiai{Đạo hàm của hàm số trên là $y'=\dfrac{2}{(2x-1)\ln 15}$.}
\end{ex}
\begin{ex}
Đạo hàm của hàm số $y=\log_{46}(-5x-5)$ là
\choice
{$y'=\dfrac{1}{x+1}$}
{$y'=\dfrac{\ln{46}}{x+1}$}
{$y'=\dfrac{1}{\ln{46}}$}
{\True $y'=\dfrac{1}{(x+1)\ln 46}$}
\loigiai{Đạo hàm của hàm số trên là $y'=\dfrac{1}{(x+1)\ln 46}$.}
\end{ex}
\begin{ex}
Đạo hàm của hàm số $y=\log_9(-4x-1)$ là
\choice
{$y'=\dfrac{4}{\ln{9}}$}
{$y'=\dfrac{4}{4x+1}$}
{$y'=\dfrac{4\ln{9}}{4x+1}$}
{\True $y'=\dfrac{4}{(4x+1)\ln 9}$}
\loigiai{Đạo hàm của hàm số trên là $y'=\dfrac{4}{(4x+1)\ln 9}$.}
\end{ex}
\begin{ex}
Đạo hàm của hàm số $y=\log_2(-3x+2)$ là
\choice
{$y'=\dfrac{3}{3x-2}$}
{$y'=\dfrac{3}{\ln{2}}$}
{$y'=\dfrac{3\ln{2}}{3x-2}$}
{\True $y'=\dfrac{3}{(3x-2)\ln 2}$}
\loigiai{Đạo hàm của hàm số trên là $y'=\dfrac{3}{(3x-2)\ln 2}$.}
\end{ex}
\begin{ex}
Đạo hàm của hàm số $y=\log_{47}(x-5)$ là
\choice
{$y'=\dfrac{\ln{47}}{x-5}$}
{\True $y'=\dfrac{1}{(x-5)\ln 47}$}
{$y'=\dfrac{1}{x-5}$}
{$y'=\dfrac{1}{\ln{47}}$}
\loigiai{Đạo hàm của hàm số trên là $y'=\dfrac{1}{(x-5)\ln 47}$.}
\end{ex}
\begin{ex}
Đạo hàm của hàm số $y=\log_{38}(-5x-5)$ là
\choice
{$y'=\dfrac{\ln{38}}{x+1}$}
{$y'=\dfrac{1}{x+1}$}
{\True $y'=\dfrac{1}{(x+1)\ln 38}$}
{$y'=\dfrac{1}{\ln{38}}$}
\loigiai{Đạo hàm của hàm số trên là $y'=\dfrac{1}{(x+1)\ln 38}$.}
\end{ex}
\begin{ex}
Đạo hàm của hàm số $y=\log_{19}(3x-1)$ là
\choice
{$y'=\dfrac{3\ln{19}}{3x-1}$}
{\True $y'=\dfrac{3}{(3x-1)\ln 19}$}
{$y'=\dfrac{3}{3x-1}$}
{$y'=\dfrac{3}{\ln{19}}$}
\loigiai{Đạo hàm của hàm số trên là $y'=\dfrac{3}{(3x-1)\ln 19}$.}
\end{ex}
\begin{ex}
Đạo hàm của hàm số $y=\log_{19}(3x+2)$ là
\choice
{$y'=\dfrac{3}{3x+2}$}
{$y'=\dfrac{3\ln{19}}{3x+2}$}
{\True $y'=\dfrac{3}{(3x+2)\ln 19}$}
{$y'=\dfrac{3}{\ln{19}}$}
\loigiai{Đạo hàm của hàm số trên là $y'=\dfrac{3}{(3x+2)\ln 19}$.}
\end{ex}
\begin{ex}
Đạo hàm của hàm số $y=\log_{11}(4x+5)$ là
\choice
{$y'=\dfrac{4}{\ln{11}}$}
{$y'=\dfrac{4}{4x+5}$}
{\True $y'=\dfrac{4}{(4x+5)\ln 11}$}
{$y'=\dfrac{4\ln{11}}{4x+5}$}
\loigiai{Đạo hàm của hàm số trên là $y'=\dfrac{4}{(4x+5)\ln 11}$.}
\end{ex}
\begin{ex}
Đạo hàm của hàm số $y=\log_{27}(-x+4)$ là
\choice
{$y'=\dfrac{1}{x-4}$}
{$y'=\dfrac{1}{\ln{27}}$}
{\True $y'=\dfrac{1}{(x-4)\ln 27}$}
{$y'=\dfrac{\ln{27}}{x-4}$}
\loigiai{Đạo hàm của hàm số trên là $y'=\dfrac{1}{(x-4)\ln 27}$.}
\end{ex}
\begin{ex}
Đạo hàm của hàm số $y=\log_{27}(x+2)$ là
\choice
{$y'=\dfrac{1}{x+2}$}
{$y'=\dfrac{1}{\ln{27}}$}
{$y'=\dfrac{\ln{27}}{x+2}$}
{\True $y'=\dfrac{1}{(x+2)\ln 27}$}
\loigiai{Đạo hàm của hàm số trên là $y'=\dfrac{1}{(x+2)\ln 27}$.}
\end{ex}
\begin{ex}
Đạo hàm của hàm số $y=\log_7(-4x+5)$ là
\choice
{$y'=\dfrac{4\ln{7}}{4x-5}$}
{$y'=\dfrac{4}{4x-5}$}
{\True $y'=\dfrac{4}{(4x-5)\ln 7}$}
{$y'=\dfrac{4}{\ln{7}}$}
\loigiai{Đạo hàm của hàm số trên là $y'=\dfrac{4}{(4x-5)\ln 7}$.}
\end{ex}
\begin{ex}
Đạo hàm của hàm số $y=\log_{24}(-x)$ là
\choice
{$y'=\dfrac{1}{x}$}
{$y'=\dfrac{\ln{24}}{x}$}
{$y'=\dfrac{1}{\ln{24}}$}
{\True $y'=\dfrac{1}{x\ln 24}$}
\loigiai{Đạo hàm của hàm số trên là $y'=\dfrac{1}{x\ln 24}$.}
\end{ex}
\begin{ex}
Đạo hàm của hàm số $y=\log_{30}(5x-4)$ là
\choice
{$y'=\dfrac{5}{5x-4}$}
{$y'=\dfrac{5}{\ln{30}}$}
{$y'=\dfrac{5\ln{30}}{5x-4}$}
{\True $y'=\dfrac{5}{(5x-4)\ln 30}$}
\loigiai{Đạo hàm của hàm số trên là $y'=\dfrac{5}{(5x-4)\ln 30}$.}
\end{ex}
\begin{ex}
Đạo hàm của hàm số $y=\log_{29}(-4x-4)$ là
\choice
{\True $y'=\dfrac{1}{(x+1)\ln 29}$}
{$y'=\dfrac{1}{\ln{29}}$}
{$y'=\dfrac{\ln{29}}{x+1}$}
{$y'=\dfrac{1}{x+1}$}
\loigiai{Đạo hàm của hàm số trên là $y'=\dfrac{1}{(x+1)\ln 29}$.}
\end{ex}
\begin{ex}
Đạo hàm của hàm số $y=\log_{44}(-5x+3)$ là
\choice
{$y'=\dfrac{5}{\ln{44}}$}
{$y'=\dfrac{5}{5x-3}$}
{$y'=\dfrac{5\ln{44}}{5x-3}$}
{\True $y'=\dfrac{5}{(5x-3)\ln 44}$}
\loigiai{Đạo hàm của hàm số trên là $y'=\dfrac{5}{(5x-3)\ln 44}$.}
\end{ex}
\begin{ex}
Đạo hàm của hàm số $y=\log_{24}(-5x+5)$ là
\choice
{\True $y'=\dfrac{1}{(x-1)\ln 24}$}
{$y'=\dfrac{1}{x-1}$}
{$y'=\dfrac{\ln{24}}{x-1}$}
{$y'=\dfrac{1}{\ln{24}}$}
\loigiai{Đạo hàm của hàm số trên là $y'=\dfrac{1}{(x-1)\ln 24}$.}
\end{ex}

\Closesolutionfile{ans}
\Closesolutionfile{ansbook}
\end{document}
\newpage
\setcounter{page}{1}
\begin{center}
{\bf\Large ĐÁP ÁN}\vspace{12pt}
\begin{multicols}{10}
\begin{Solution}{1}
B
\end{Solution}
\begin{Solution}{2}
B
\end{Solution}
\begin{Solution}{3}
C
\end{Solution}
\begin{Solution}{4}
C
\end{Solution}
\begin{Solution}{5}
A
\end{Solution}
\begin{Solution}{6}
C
\end{Solution}
\begin{Solution}{7}
B
\end{Solution}
\begin{Solution}{8}
B
\end{Solution}
\begin{Solution}{9}
D
\end{Solution}
\begin{Solution}{10}
A
\end{Solution}
\begin{Solution}{11}
C
\end{Solution}
\begin{Solution}{12}
A
\end{Solution}
\begin{Solution}{13}
A
\end{Solution}
\begin{Solution}{14}
B
\end{Solution}
\begin{Solution}{15}
B
\end{Solution}
\begin{Solution}{16}
C
\end{Solution}
\begin{Solution}{17}
B
\end{Solution}
\begin{Solution}{18}
C
\end{Solution}
\begin{Solution}{19}
C
\end{Solution}
\begin{Solution}{20}
A
\end{Solution}
\begin{Solution}{21}
D
\end{Solution}
\begin{Solution}{22}
C
\end{Solution}
\begin{Solution}{23}
B
\end{Solution}
\begin{Solution}{24}
C
\end{Solution}
\begin{Solution}{25}
C
\end{Solution}
\begin{Solution}{26}
C
\end{Solution}
\begin{Solution}{27}
B
\end{Solution}
\begin{Solution}{28}
B
\end{Solution}
\begin{Solution}{29}
D
\end{Solution}
\begin{Solution}{30}
C
\end{Solution}
\begin{Solution}{31}
D
\end{Solution}
\begin{Solution}{32}
B
\end{Solution}
\begin{Solution}{33}
A
\end{Solution}
\begin{Solution}{34}
C
\end{Solution}
\begin{Solution}{35}
A
\end{Solution}
\begin{Solution}{36}
A
\end{Solution}
\begin{Solution}{37}
B
\end{Solution}
\begin{Solution}{38}
C
\end{Solution}
\begin{Solution}{39}
B
\end{Solution}
\begin{Solution}{40}
D
\end{Solution}
\begin{Solution}{41}
A
\end{Solution}
\begin{Solution}{42}
C
\end{Solution}
\begin{Solution}{43}
B
\end{Solution}
\begin{Solution}{44}
C
\end{Solution}
\begin{Solution}{45}
D
\end{Solution}
\begin{Solution}{46}
A
\end{Solution}
\begin{Solution}{47}
A
\end{Solution}
\begin{Solution}{48}
A
\end{Solution}
\begin{Solution}{49}
A
\end{Solution}
\begin{Solution}{50}
B
\end{Solution}
\begin{Solution}{51}
C
\end{Solution}
\begin{Solution}{52}
C
\end{Solution}
\begin{Solution}{53}
A
\end{Solution}
\begin{Solution}{54}
A
\end{Solution}
\begin{Solution}{55}
D
\end{Solution}
\begin{Solution}{56}
D
\end{Solution}
\begin{Solution}{57}
C
\end{Solution}
\begin{Solution}{58}
C
\end{Solution}
\begin{Solution}{59}
A
\end{Solution}
\begin{Solution}{60}
B
\end{Solution}
\begin{Solution}{61}
C
\end{Solution}
\begin{Solution}{62}
B
\end{Solution}
\begin{Solution}{63}
D
\end{Solution}
\begin{Solution}{64}
A
\end{Solution}
\begin{Solution}{65}
D
\end{Solution}
\begin{Solution}{66}
B
\end{Solution}
\begin{Solution}{67}
A
\end{Solution}
\begin{Solution}{68}
B
\end{Solution}
\begin{Solution}{69}
B
\end{Solution}
\begin{Solution}{70}
A
\end{Solution}
\begin{Solution}{71}
D
\end{Solution}
\begin{Solution}{72}
B
\end{Solution}
\begin{Solution}{73}
C
\end{Solution}
\begin{Solution}{74}
D
\end{Solution}
\begin{Solution}{75}
D
\end{Solution}
\begin{Solution}{76}
C
\end{Solution}
\begin{Solution}{77}
C
\end{Solution}
\begin{Solution}{78}
A
\end{Solution}
\begin{Solution}{79}
A
\end{Solution}
\begin{Solution}{80}
C
\end{Solution}
\begin{Solution}{81}
A
\end{Solution}
\begin{Solution}{82}
C
\end{Solution}
\begin{Solution}{83}
A
\end{Solution}
\begin{Solution}{84}
D
\end{Solution}
\begin{Solution}{85}
D
\end{Solution}
\begin{Solution}{86}
D
\end{Solution}
\begin{Solution}{87}
D
\end{Solution}
\begin{Solution}{88}
B
\end{Solution}
\begin{Solution}{89}
C
\end{Solution}
\begin{Solution}{90}
B
\end{Solution}
\begin{Solution}{91}
C
\end{Solution}
\begin{Solution}{92}
C
\end{Solution}
\begin{Solution}{93}
C
\end{Solution}
\begin{Solution}{94}
D
\end{Solution}
\begin{Solution}{95}
C
\end{Solution}
\begin{Solution}{96}
D
\end{Solution}
\begin{Solution}{97}
D
\end{Solution}
\begin{Solution}{98}
A
\end{Solution}
\begin{Solution}{99}
D
\end{Solution}
\begin{Solution}{100}
A
\end{Solution}

\end{multicols}
\end{center}
\newpage
\setcounter{page}{1}
\begin{center}
{\bf\Large LỜI GIẢI CHI TIẾT}
\end{center}
\input{ans/LG}
\end{document}