\begin{ex}
Trong mặt phẳng $Oxy$, cho phương trình đường tròn $(x -6)^2   + (y + 6)^2 = 1$. Tọa độ tâm của đường tròn là
\choice
{ \True $I(6, -6)$ }
{ $I(6, 6)$ }
{ $I(-6, 6)$ }
{ $I(-6, -6)$ }
\end{ex}

\begin{ex}
Trong mặt phẳng $Oxy$, cho phương trình đường tròn $(x -3)^2   + (y -2)^2 = 4$. Tọa độ tâm của đường tròn là
\choice
{ $I(-2, 3)$ }
{ $I(2, -3)$ }
{ $I(-3, -2)$ }
{ \True $I(3, 2)$ }
\end{ex}

\begin{ex}
Trong mặt phẳng $Oxy$, cho phương trình đường tròn $(x -6)^2   + (y -7)^2 = 49$. Tọa độ tâm của đường tròn là
\choice
{ $I(-6, -7)$ }
{ $I(7, -6)$ }
{ $I(-7, 6)$ }
{ \True $I(6, 7)$ }
\end{ex}

\begin{ex}
Trong mặt phẳng $Oxy$, cho phương trình đường tròn $(x -7)^2   + (y -3)^2 = 9$. Tọa độ tâm của đường tròn là
\choice
{ \True $I(7, 3)$ }
{ $I(-7, -3)$ }
{ $I(-3, 7)$ }
{ $I(3, -7)$ }
\end{ex}

\begin{ex}
Trong mặt phẳng $Oxy$, cho phương trình đường tròn $x^2 + (y -3)^2 = 25$. Tọa độ tâm của đường tròn là
\choice
{ \True $I(0, 3)$ }
{ $I(3, 0)$ }
{ $I(-3, 0)$ }
{ $I(0, -3)$ }
\end{ex}

\begin{ex}
Trong mặt phẳng $Oxy$, cho phương trình đường tròn $(x -2)^2   + (y + 3)^2 = 36$. Tọa độ tâm của đường tròn là
\choice
{ $I(-3, -2)$ }
{ $I(3, 2)$ }
{ \True $I(2, -3)$ }
{ $I(-2, 3)$ }
\end{ex}

\begin{ex}
Trong mặt phẳng $Oxy$, cho phương trình đường tròn $(x -2)^2   + (y + 4)^2 = 36$. Tọa độ tâm của đường tròn là
\choice
{ $I(4, 2)$ }
{ $I(-4, -2)$ }
{ \True $I(2, -4)$ }
{ $I(-2, 4)$ }
\end{ex}

\begin{ex}
Trong mặt phẳng $Oxy$, cho phương trình đường tròn $(x -6)^2   + (y -5)^2 = 25$. Tọa độ tâm của đường tròn là
\choice
{ $I(5, -6)$ }
{ $I(-5, 6)$ }
{ \True $I(6, 5)$ }
{ $I(-6, -5)$ }
\end{ex}

\begin{ex}
Trong mặt phẳng $Oxy$, cho phương trình đường tròn $(x + 6)^2  + (y -2)^2 = 16$. Tọa độ tâm của đường tròn là
\choice
{ $I(-2, -6)$ }
{ $I(6, -2)$ }
{ \True $I(-6, 2)$ }
{ $I(2, 6)$ }
\end{ex}

\begin{ex}
Trong mặt phẳng $Oxy$, cho phương trình đường tròn $(x -4)^2   + (y -7)^2 = 9$. Tọa độ tâm của đường tròn là
\choice
{ \True $I(4, 7)$ }
{ $I(7, -4)$ }
{ $I(-7, 4)$ }
{ $I(-4, -7)$ }
\end{ex}

\begin{ex}
Trong mặt phẳng $Oxy$, cho phương trình đường tròn $(x -1)^2   + (y -1)^2 = 25$. Tọa độ tâm của đường tròn là
\choice
{ $I(-1, -1)$ }
{ $I(-1, 1)$ }
{ \True $I(1, 1)$ }
{ $I(1, -1)$ }
\end{ex}

\begin{ex}
Trong mặt phẳng $Oxy$, cho phương trình đường tròn $(x + 5)^2  + (y -5)^2 = 1$. Tọa độ tâm của đường tròn là
\choice
{ \True $I(-5, 5)$ }
{ $I(5, 5)$ }
{ $I(5, -5)$ }
{ $I(-5, -5)$ }
\end{ex}

\begin{ex}
Trong mặt phẳng $Oxy$, cho phương trình đường tròn $(x -5)^2   + (y -4)^2 = 36$. Tọa độ tâm của đường tròn là
\choice
{ $I(4, -5)$ }
{ $I(-5, -4)$ }
{ $I(-4, 5)$ }
{ \True $I(5, 4)$ }
\end{ex}

\begin{ex}
Trong mặt phẳng $Oxy$, cho phương trình đường tròn $(x -2)^2   + (y + 1)^2 = 25$. Tọa độ tâm của đường tròn là
\choice
{ $I(-2, 1)$ }
{ $I(1, 2)$ }
{ $I(-1, -2)$ }
{ \True $I(2, -1)$ }
\end{ex}

\begin{ex}
Trong mặt phẳng $Oxy$, cho phương trình đường tròn $(x -6)^2   + (y -6)^2 = 16$. Tọa độ tâm của đường tròn là
\choice
{ $I(6, -6)$ }
{ $I(-6, 6)$ }
{ \True $I(6, 6)$ }
{ $I(-6, -6)$ }
\end{ex}

\begin{ex}
Trong mặt phẳng $Oxy$, cho phương trình đường tròn $(x -4)^2   + (y -3)^2 = 16$. Tọa độ tâm của đường tròn là
\choice
{ \True $I(4, 3)$ }
{ $I(3, -4)$ }
{ $I(-4, -3)$ }
{ $I(-3, 4)$ }
\end{ex}

\begin{ex}
Trong mặt phẳng $Oxy$, cho phương trình đường tròn $(x -4)^2   + (y -4)^2 = 1$. Tọa độ tâm của đường tròn là
\choice
{ $I(4, -4)$ }
{ $I(-4, -4)$ }
{ $I(-4, 4)$ }
{ \True $I(4, 4)$ }
\end{ex}

\begin{ex}
Trong mặt phẳng $Oxy$, cho phương trình đường tròn $(x + 6)^2  + (y -4)^2 = 4$. Tọa độ tâm của đường tròn là
\choice
{ $I(-4, -6)$ }
{ \True $I(-6, 4)$ }
{ $I(6, -4)$ }
{ $I(4, 6)$ }
\end{ex}

\begin{ex}
Trong mặt phẳng $Oxy$, cho phương trình đường tròn $(x -3)^2   + (y -1)^2 = 9$. Tọa độ tâm của đường tròn là
\choice
{ $I(-1, 3)$ }
{ $I(-3, -1)$ }
{ \True $I(3, 1)$ }
{ $I(1, -3)$ }
\end{ex}

\begin{ex}
Trong mặt phẳng $Oxy$, cho phương trình đường tròn $(x -6)^2   + (y + 5)^2 = 49$. Tọa độ tâm của đường tròn là
\choice
{ $I(5, 6)$ }
{ \True $I(6, -5)$ }
{ $I(-5, -6)$ }
{ $I(-6, 5)$ }
\end{ex}

\begin{ex}
Trong mặt phẳng $Oxy$, cho phương trình đường tròn $(x + 1)^2  + (y + 4)^2 = 9$. Tọa độ tâm của đường tròn là
\choice
{ $I(-4, 1)$ }
{ \True $I(-1, -4)$ }
{ $I(4, -1)$ }
{ $I(1, 4)$ }
\end{ex}

\begin{ex}
Trong mặt phẳng $Oxy$, cho phương trình đường tròn $(x -5)^2   + (y -2)^2 = 1$. Tọa độ tâm của đường tròn là
\choice
{ \True $I(5, 2)$ }
{ $I(-2, 5)$ }
{ $I(2, -5)$ }
{ $I(-5, -2)$ }
\end{ex}

\begin{ex}
Trong mặt phẳng $Oxy$, cho phương trình đường tròn $(x -4)^2   + (y + 4)^2 = 49$. Tọa độ tâm của đường tròn là
\choice
{ $I(-4, -4)$ }
{ \True $I(4, -4)$ }
{ $I(4, 4)$ }
{ $I(-4, 4)$ }
\end{ex}

\begin{ex}
Trong mặt phẳng $Oxy$, cho phương trình đường tròn $(x + 6)^2  + (y -3)^2 = 49$. Tọa độ tâm của đường tròn là
\choice
{ \True $I(-6, 3)$ }
{ $I(-3, -6)$ }
{ $I(3, 6)$ }
{ $I(6, -3)$ }
\end{ex}

\begin{ex}
Trong mặt phẳng $Oxy$, cho phương trình đường tròn $(x -3)^2   + (y -4)^2 = 1$. Tọa độ tâm của đường tròn là
\choice
{ $I(-3, -4)$ }
{ $I(4, -3)$ }
{ \True $I(3, 4)$ }
{ $I(-4, 3)$ }
\end{ex}

\begin{ex}
Trong mặt phẳng $Oxy$, cho phương trình đường tròn $(x -2)^2   + (y -5)^2 = 1$. Tọa độ tâm của đường tròn là
\choice
{ $I(-5, 2)$ }
{ $I(5, -2)$ }
{ $I(-2, -5)$ }
{ \True $I(2, 5)$ }
\end{ex}

\begin{ex}
Trong mặt phẳng $Oxy$, cho phương trình đường tròn $(x + 1)^2  + (y -4)^2 = 16$. Tọa độ tâm của đường tròn là
\choice
{ $I(-4, -1)$ }
{ \True $I(-1, 4)$ }
{ $I(1, -4)$ }
{ $I(4, 1)$ }
\end{ex}

\begin{ex}
Trong mặt phẳng $Oxy$, cho phương trình đường tròn $(x + 2)^2  + (y + 1)^2 = 4$. Tọa độ tâm của đường tròn là
\choice
{ $I(-1, 2)$ }
{ $I(1, -2)$ }
{ $I(2, 1)$ }
{ \True $I(-2, -1)$ }
\end{ex}

\begin{ex}
Trong mặt phẳng $Oxy$, cho phương trình đường tròn $(x -2)^2   + (y -3)^2 = 16$. Tọa độ tâm của đường tròn là
\choice
{ $I(3, -2)$ }
{ $I(-2, -3)$ }
{ \True $I(2, 3)$ }
{ $I(-3, 2)$ }
\end{ex}

\begin{ex}
Trong mặt phẳng $Oxy$, cho phương trình đường tròn $(x + 4)^2  + (y + 3)^2 = 25$. Tọa độ tâm của đường tròn là
\choice
{ $I(4, 3)$ }
{ $I(3, -4)$ }
{ \True $I(-4, -3)$ }
{ $I(-3, 4)$ }
\end{ex}

\begin{ex}
Trong mặt phẳng $Oxy$, cho phương trình đường tròn $(x -7)^2   + (y -7)^2 = 49$. Tọa độ tâm của đường tròn là
\choice
{ $I(-7, 7)$ }
{ $I(7, -7)$ }
{ $I(-7, -7)$ }
{ \True $I(7, 7)$ }
\end{ex}

\begin{ex}
Trong mặt phẳng $Oxy$, cho phương trình đường tròn $(x + 6)^2  + (y + 6)^2 = 49$. Tọa độ tâm của đường tròn là
\choice
{ \True $I(-6, -6)$ }
{ $I(6, 6)$ }
{ $I(6, -6)$ }
{ $I(-6, 6)$ }
\end{ex}

\begin{ex}
Trong mặt phẳng $Oxy$, cho phương trình đường tròn $(x -7)^2   + (y -3)^2 = 4$. Tọa độ tâm của đường tròn là
\choice
{ \True $I(7, 3)$ }
{ $I(3, -7)$ }
{ $I(-3, 7)$ }
{ $I(-7, -3)$ }
\end{ex}

\begin{ex}
Trong mặt phẳng $Oxy$, cho phương trình đường tròn $(x -6)^2   + (y -7)^2 = 16$. Tọa độ tâm của đường tròn là
\choice
{ \True $I(6, 7)$ }
{ $I(7, -6)$ }
{ $I(-7, 6)$ }
{ $I(-6, -7)$ }
\end{ex}

\begin{ex}
Trong mặt phẳng $Oxy$, cho phương trình đường tròn $(x -4)^2   + (y -1)^2 = 25$. Tọa độ tâm của đường tròn là
\choice
{ $I(-1, 4)$ }
{ \True $I(4, 1)$ }
{ $I(1, -4)$ }
{ $I(-4, -1)$ }
\end{ex}

\begin{ex}
Trong mặt phẳng $Oxy$, cho phương trình đường tròn $(x + 3)^2  + (y + 5)^2 = 49$. Tọa độ tâm của đường tròn là
\choice
{ $I(3, 5)$ }
{ $I(5, -3)$ }
{ $I(-5, 3)$ }
{ \True $I(-3, -5)$ }
\end{ex}

\begin{ex}
Trong mặt phẳng $Oxy$, cho phương trình đường tròn $(x -4)^2   + (y -3)^2 = 25$. Tọa độ tâm của đường tròn là
\choice
{ $I(3, -4)$ }
{ \True $I(4, 3)$ }
{ $I(-3, 4)$ }
{ $I(-4, -3)$ }
\end{ex}

\begin{ex}
Trong mặt phẳng $Oxy$, cho phương trình đường tròn $(x -2)^2   + (y -6)^2 = 36$. Tọa độ tâm của đường tròn là
\choice
{ $I(-6, 2)$ }
{ \True $I(2, 6)$ }
{ $I(6, -2)$ }
{ $I(-2, -6)$ }
\end{ex}

\begin{ex}
Trong mặt phẳng $Oxy$, cho phương trình đường tròn $(x -4)^2   + (y -1)^2 = 36$. Tọa độ tâm của đường tròn là
\choice
{ $I(1, -4)$ }
{ \True $I(4, 1)$ }
{ $I(-4, -1)$ }
{ $I(-1, 4)$ }
\end{ex}

\begin{ex}
Trong mặt phẳng $Oxy$, cho phương trình đường tròn $(x -5)^2   + (y -7)^2 = 1$. Tọa độ tâm của đường tròn là
\choice
{ $I(7, -5)$ }
{ \True $I(5, 7)$ }
{ $I(-5, -7)$ }
{ $I(-7, 5)$ }
\end{ex}

\begin{ex}
Trong mặt phẳng $Oxy$, cho phương trình đường tròn $x^2 + (y -2)^2 = 4$. Tọa độ tâm của đường tròn là
\choice
{ \True $I(0, 2)$ }
{ $I(0, -2)$ }
{ $I(-2, 0)$ }
{ $I(2, 0)$ }
\end{ex}

\begin{ex}
Trong mặt phẳng $Oxy$, cho phương trình đường tròn $(x + 3)^2  + (y + 4)^2 = 9$. Tọa độ tâm của đường tròn là
\choice
{ $I(3, 4)$ }
{ $I(4, -3)$ }
{ $I(-4, 3)$ }
{ \True $I(-3, -4)$ }
\end{ex}

\begin{ex}
Trong mặt phẳng $Oxy$, cho phương trình đường tròn $(x + 1)^2  + (y -4)^2 = 16$. Tọa độ tâm của đường tròn là
\choice
{ $I(-4, -1)$ }
{ $I(1, -4)$ }
{ \True $I(-1, 4)$ }
{ $I(4, 1)$ }
\end{ex}

\begin{ex}
Trong mặt phẳng $Oxy$, cho phương trình đường tròn $(x -4)^2   + (y -6)^2 = 16$. Tọa độ tâm của đường tròn là
\choice
{ $I(-6, 4)$ }
{ \True $I(4, 6)$ }
{ $I(-4, -6)$ }
{ $I(6, -4)$ }
\end{ex}

\begin{ex}
Trong mặt phẳng $Oxy$, cho phương trình đường tròn $(x + 5)^2  + (y + 4)^2 = 9$. Tọa độ tâm của đường tròn là
\choice
{ \True $I(-5, -4)$ }
{ $I(-4, 5)$ }
{ $I(5, 4)$ }
{ $I(4, -5)$ }
\end{ex}

\begin{ex}
Trong mặt phẳng $Oxy$, cho phương trình đường tròn $(x + 3)^2  + (y -5)^2 = 25$. Tọa độ tâm của đường tròn là
\choice
{ \True $I(-3, 5)$ }
{ $I(5, 3)$ }
{ $I(3, -5)$ }
{ $I(-5, -3)$ }
\end{ex}

\begin{ex}
Trong mặt phẳng $Oxy$, cho phương trình đường tròn $(x -6)^2   + (y + 4)^2 = 16$. Tọa độ tâm của đường tròn là
\choice
{ \True $I(6, -4)$ }
{ $I(4, 6)$ }
{ $I(-6, 4)$ }
{ $I(-4, -6)$ }
\end{ex}

\begin{ex}
Trong mặt phẳng $Oxy$, cho phương trình đường tròn $x^2 + (y -3)^2 = 1$. Tọa độ tâm của đường tròn là
\choice
{ \True $I(0, 3)$ }
{ $I(-3, 0)$ }
{ $I(3, 0)$ }
{ $I(0, -3)$ }
\end{ex}

\begin{ex}
Trong mặt phẳng $Oxy$, cho phương trình đường tròn $x^2 + (y -2)^2 = 9$. Tọa độ tâm của đường tròn là
\choice
{ $I(2, 0)$ }
{ $I(0, -2)$ }
{ \True $I(0, 2)$ }
{ $I(-2, 0)$ }
\end{ex}

\begin{ex}
Trong mặt phẳng $Oxy$, cho phương trình đường tròn $(x + 1)^2  + (y -2)^2 = 9$. Tọa độ tâm của đường tròn là
\choice
{ $I(2, 1)$ }
{ $I(1, -2)$ }
{ \True $I(-1, 2)$ }
{ $I(-2, -1)$ }
\end{ex}

\begin{ex}
Trong mặt phẳng $Oxy$, cho phương trình đường tròn $(x -4)^2   + (y + 3)^2 = 4$. Tọa độ tâm của đường tròn là
\choice
{ $I(-3, -4)$ }
{ $I(-4, 3)$ }
{ \True $I(4, -3)$ }
{ $I(3, 4)$ }
\end{ex}

\begin{ex}
Trong mặt phẳng $Oxy$, cho phương trình đường tròn $(x -1)^2   + (y -7)^2 = 4$. Tọa độ tâm của đường tròn là
\choice
{ $I(7, -1)$ }
{ \True $I(1, 7)$ }
{ $I(-7, 1)$ }
{ $I(-1, -7)$ }
\end{ex}

\begin{ex}
Trong mặt phẳng $Oxy$, cho phương trình đường tròn $(x -6)^2   + (y -6)^2 = 1$. Tọa độ tâm của đường tròn là
\choice
{ $I(-6, -6)$ }
{ $I(-6, 6)$ }
{ \True $I(6, 6)$ }
{ $I(6, -6)$ }
\end{ex}

\begin{ex}
Trong mặt phẳng $Oxy$, cho phương trình đường tròn $(x + 5)^2  + (y -6)^2 = 16$. Tọa độ tâm của đường tròn là
\choice
{ \True $I(-5, 6)$ }
{ $I(5, -6)$ }
{ $I(-6, -5)$ }
{ $I(6, 5)$ }
\end{ex}

\begin{ex}
Trong mặt phẳng $Oxy$, cho phương trình đường tròn $(x + 5)^2  + (y -5)^2 = 36$. Tọa độ tâm của đường tròn là
\choice
{ \True $I(-5, 5)$ }
{ $I(-5, -5)$ }
{ $I(5, -5)$ }
{ $I(5, 5)$ }
\end{ex}

\begin{ex}
Trong mặt phẳng $Oxy$, cho phương trình đường tròn $(x -5)^2   + (y + 4)^2 = 9$. Tọa độ tâm của đường tròn là
\choice
{ $I(-5, 4)$ }
{ $I(4, 5)$ }
{ $I(-4, -5)$ }
{ \True $I(5, -4)$ }
\end{ex}

\begin{ex}
Trong mặt phẳng $Oxy$, cho phương trình đường tròn $(x -4)^2   + (y -5)^2 = 25$. Tọa độ tâm của đường tròn là
\choice
{ \True $I(4, 5)$ }
{ $I(-4, -5)$ }
{ $I(-5, 4)$ }
{ $I(5, -4)$ }
\end{ex}

\begin{ex}
Trong mặt phẳng $Oxy$, cho phương trình đường tròn $(x + 5)^2  + (y -5)^2 = 9$. Tọa độ tâm của đường tròn là
\choice
{ $I(5, 5)$ }
{ $I(-5, -5)$ }
{ $I(5, -5)$ }
{ \True $I(-5, 5)$ }
\end{ex}

\begin{ex}
Trong mặt phẳng $Oxy$, cho phương trình đường tròn $(x -1)^2   + (y -5)^2 = 49$. Tọa độ tâm của đường tròn là
\choice
{ \True $I(1, 5)$ }
{ $I(-5, 1)$ }
{ $I(5, -1)$ }
{ $I(-1, -5)$ }
\end{ex}

\begin{ex}
Trong mặt phẳng $Oxy$, cho phương trình đường tròn $(x -3)^2   + (y + 6)^2 = 4$. Tọa độ tâm của đường tròn là
\choice
{ \True $I(3, -6)$ }
{ $I(-6, -3)$ }
{ $I(-3, 6)$ }
{ $I(6, 3)$ }
\end{ex}

\begin{ex}
Trong mặt phẳng $Oxy$, cho phương trình đường tròn $(x -1)^2   + (y -7)^2 = 16$. Tọa độ tâm của đường tròn là
\choice
{ $I(-7, 1)$ }
{ $I(-1, -7)$ }
{ \True $I(1, 7)$ }
{ $I(7, -1)$ }
\end{ex}

\begin{ex}
Trong mặt phẳng $Oxy$, cho phương trình đường tròn $(x -2)^2   + (y + 5)^2 = 25$. Tọa độ tâm của đường tròn là
\choice
{ $I(-2, 5)$ }
{ $I(5, 2)$ }
{ $I(-5, -2)$ }
{ \True $I(2, -5)$ }
\end{ex}

\begin{ex}
Trong mặt phẳng $Oxy$, cho phương trình đường tròn $(x -1)^2   + (y -2)^2 = 4$. Tọa độ tâm của đường tròn là
\choice
{ $I(-1, -2)$ }
{ \True $I(1, 2)$ }
{ $I(-2, 1)$ }
{ $I(2, -1)$ }
\end{ex}

\begin{ex}
Trong mặt phẳng $Oxy$, cho phương trình đường tròn $(x + 3)^2  + (y + 3)^2 = 16$. Tọa độ tâm của đường tròn là
\choice
{ $I(3, -3)$ }
{ \True $I(-3, -3)$ }
{ $I(3, 3)$ }
{ $I(-3, 3)$ }
\end{ex}

\begin{ex}
Trong mặt phẳng $Oxy$, cho phương trình đường tròn $(x -1)^2   + (y + 2)^2 = 36$. Tọa độ tâm của đường tròn là
\choice
{ $I(2, 1)$ }
{ \True $I(1, -2)$ }
{ $I(-2, -1)$ }
{ $I(-1, 2)$ }
\end{ex}

\begin{ex}
Trong mặt phẳng $Oxy$, cho phương trình đường tròn $(x -1)^2   + (y + 6)^2 = 36$. Tọa độ tâm của đường tròn là
\choice
{ $I(-1, 6)$ }
{ $I(6, 1)$ }
{ $I(-6, -1)$ }
{ \True $I(1, -6)$ }
\end{ex}

\begin{ex}
Trong mặt phẳng $Oxy$, cho phương trình đường tròn $(x + 2)^2  + (y + 6)^2 = 25$. Tọa độ tâm của đường tròn là
\choice
{ $I(2, 6)$ }
{ \True $I(-2, -6)$ }
{ $I(6, -2)$ }
{ $I(-6, 2)$ }
\end{ex}

\begin{ex}
Trong mặt phẳng $Oxy$, cho phương trình đường tròn $(x -4)^2   + (y -1)^2 = 16$. Tọa độ tâm của đường tròn là
\choice
{ $I(-1, 4)$ }
{ \True $I(4, 1)$ }
{ $I(-4, -1)$ }
{ $I(1, -4)$ }
\end{ex}

\begin{ex}
Trong mặt phẳng $Oxy$, cho phương trình đường tròn $(x -3)^2   + (y + 1)^2 = 1$. Tọa độ tâm của đường tròn là
\choice
{ $I(1, 3)$ }
{ \True $I(3, -1)$ }
{ $I(-1, -3)$ }
{ $I(-3, 1)$ }
\end{ex}

\begin{ex}
Trong mặt phẳng $Oxy$, cho phương trình đường tròn $(x + 5)^2  + (y + 1)^2 = 25$. Tọa độ tâm của đường tròn là
\choice
{ $I(1, -5)$ }
{ \True $I(-5, -1)$ }
{ $I(-1, 5)$ }
{ $I(5, 1)$ }
\end{ex}

\begin{ex}
Trong mặt phẳng $Oxy$, cho phương trình đường tròn $(x -7)^2   + (y -6)^2 = 1$. Tọa độ tâm của đường tròn là
\choice
{ \True $I(7, 6)$ }
{ $I(-7, -6)$ }
{ $I(6, -7)$ }
{ $I(-6, 7)$ }
\end{ex}

\begin{ex}
Trong mặt phẳng $Oxy$, cho phương trình đường tròn $(x -5)^2   + (y -5)^2 = 1$. Tọa độ tâm của đường tròn là
\choice
{ $I(5, -5)$ }
{ \True $I(5, 5)$ }
{ $I(-5, -5)$ }
{ $I(-5, 5)$ }
\end{ex}

\begin{ex}
Trong mặt phẳng $Oxy$, cho phương trình đường tròn $(x -4)^2   + (y + 6)^2 = 49$. Tọa độ tâm của đường tròn là
\choice
{ \True $I(4, -6)$ }
{ $I(6, 4)$ }
{ $I(-6, -4)$ }
{ $I(-4, 6)$ }
\end{ex}

\begin{ex}
Trong mặt phẳng $Oxy$, cho phương trình đường tròn $(x + 4)^2  + (y -7)^2 = 36$. Tọa độ tâm của đường tròn là
\choice
{ \True $I(-4, 7)$ }
{ $I(-7, -4)$ }
{ $I(4, -7)$ }
{ $I(7, 4)$ }
\end{ex}

\begin{ex}
Trong mặt phẳng $Oxy$, cho phương trình đường tròn $(x -2)^2   + (y -7)^2 = 16$. Tọa độ tâm của đường tròn là
\choice
{ \True $I(2, 7)$ }
{ $I(7, -2)$ }
{ $I(-7, 2)$ }
{ $I(-2, -7)$ }
\end{ex}

\begin{ex}
Trong mặt phẳng $Oxy$, cho phương trình đường tròn $(x + 1)^2  + y^2 = 16$. Tọa độ tâm của đường tròn là
\choice
{ $I(0, 1)$ }
{ $I(1, 0)$ }
{ \True $I(-1, 0)$ }
{ $I(0, -1)$ }
\end{ex}

\begin{ex}
Trong mặt phẳng $Oxy$, cho phương trình đường tròn $(x + 3)^2  + (y + 3)^2 = 25$. Tọa độ tâm của đường tròn là
\choice
{ $I(3, 3)$ }
{ $I(3, -3)$ }
{ $I(-3, 3)$ }
{ \True $I(-3, -3)$ }
\end{ex}

\begin{ex}
Trong mặt phẳng $Oxy$, cho phương trình đường tròn $(x + 2)^2  + (y -7)^2 = 25$. Tọa độ tâm của đường tròn là
\choice
{ $I(2, -7)$ }
{ $I(7, 2)$ }
{ \True $I(-2, 7)$ }
{ $I(-7, -2)$ }
\end{ex}

\begin{ex}
Trong mặt phẳng $Oxy$, cho phương trình đường tròn $(x -6)^2   + (y + 6)^2 = 9$. Tọa độ tâm của đường tròn là
\choice
{ $I(-6, 6)$ }
{ \True $I(6, -6)$ }
{ $I(6, 6)$ }
{ $I(-6, -6)$ }
\end{ex}

\begin{ex}
Trong mặt phẳng $Oxy$, cho phương trình đường tròn $(x -5)^2   + (y + 2)^2 = 1$. Tọa độ tâm của đường tròn là
\choice
{ \True $I(5, -2)$ }
{ $I(-2, -5)$ }
{ $I(2, 5)$ }
{ $I(-5, 2)$ }
\end{ex}

\begin{ex}
Trong mặt phẳng $Oxy$, cho phương trình đường tròn $(x -5)^2   + (y -7)^2 = 16$. Tọa độ tâm của đường tròn là
\choice
{ $I(7, -5)$ }
{ $I(-5, -7)$ }
{ $I(-7, 5)$ }
{ \True $I(5, 7)$ }
\end{ex}

\begin{ex}
Trong mặt phẳng $Oxy$, cho phương trình đường tròn $(x -7)^2   + (y + 2)^2 = 4$. Tọa độ tâm của đường tròn là
\choice
{ $I(-7, 2)$ }
{ \True $I(7, -2)$ }
{ $I(2, 7)$ }
{ $I(-2, -7)$ }
\end{ex}

\begin{ex}
Trong mặt phẳng $Oxy$, cho phương trình đường tròn $(x -1)^2   + (y -6)^2 = 16$. Tọa độ tâm của đường tròn là
\choice
{ $I(-1, -6)$ }
{ $I(-6, 1)$ }
{ \True $I(1, 6)$ }
{ $I(6, -1)$ }
\end{ex}

\begin{ex}
Trong mặt phẳng $Oxy$, cho phương trình đường tròn $(x -2)^2   + (y -2)^2 = 1$. Tọa độ tâm của đường tròn là
\choice
{ $I(-2, 2)$ }
{ \True $I(2, 2)$ }
{ $I(-2, -2)$ }
{ $I(2, -2)$ }
\end{ex}

\begin{ex}
Trong mặt phẳng $Oxy$, cho phương trình đường tròn $(x + 5)^2  + (y -4)^2 = 16$. Tọa độ tâm của đường tròn là
\choice
{ $I(4, 5)$ }
{ \True $I(-5, 4)$ }
{ $I(-4, -5)$ }
{ $I(5, -4)$ }
\end{ex}

\begin{ex}
Trong mặt phẳng $Oxy$, cho phương trình đường tròn $(x -2)^2   + (y -6)^2 = 1$. Tọa độ tâm của đường tròn là
\choice
{ $I(-6, 2)$ }
{ \True $I(2, 6)$ }
{ $I(6, -2)$ }
{ $I(-2, -6)$ }
\end{ex}

\begin{ex}
Trong mặt phẳng $Oxy$, cho phương trình đường tròn $(x + 3)^2  + (y -4)^2 = 36$. Tọa độ tâm của đường tròn là
\choice
{ $I(4, 3)$ }
{ $I(-4, -3)$ }
{ \True $I(-3, 4)$ }
{ $I(3, -4)$ }
\end{ex}

\begin{ex}
Trong mặt phẳng $Oxy$, cho phương trình đường tròn $(x + 6)^2  + (y -4)^2 = 16$. Tọa độ tâm của đường tròn là
\choice
{ $I(4, 6)$ }
{ $I(-4, -6)$ }
{ \True $I(-6, 4)$ }
{ $I(6, -4)$ }
\end{ex}

\begin{ex}
Trong mặt phẳng $Oxy$, cho phương trình đường tròn $(x -6)^2   + (y + 2)^2 = 9$. Tọa độ tâm của đường tròn là
\choice
{ $I(-6, 2)$ }
{ \True $I(6, -2)$ }
{ $I(-2, -6)$ }
{ $I(2, 6)$ }
\end{ex}

\begin{ex}
Trong mặt phẳng $Oxy$, cho phương trình đường tròn $(x + 5)^2  + (y + 3)^2 = 25$. Tọa độ tâm của đường tròn là
\choice
{ $I(5, 3)$ }
{ $I(-3, 5)$ }
{ \True $I(-5, -3)$ }
{ $I(3, -5)$ }
\end{ex}

\begin{ex}
Trong mặt phẳng $Oxy$, cho phương trình đường tròn $(x + 4)^2  + (y + 2)^2 = 25$. Tọa độ tâm của đường tròn là
\choice
{ $I(2, -4)$ }
{ \True $I(-4, -2)$ }
{ $I(-2, 4)$ }
{ $I(4, 2)$ }
\end{ex}

\begin{ex}
Trong mặt phẳng $Oxy$, cho phương trình đường tròn $(x + 5)^2  + (y -3)^2 = 9$. Tọa độ tâm của đường tròn là
\choice
{ \True $I(-5, 3)$ }
{ $I(-3, -5)$ }
{ $I(3, 5)$ }
{ $I(5, -3)$ }
\end{ex}

\begin{ex}
Trong mặt phẳng $Oxy$, cho phương trình đường tròn $(x + 3)^2  + (y + 4)^2 = 25$. Tọa độ tâm của đường tròn là
\choice
{ \True $I(-3, -4)$ }
{ $I(3, 4)$ }
{ $I(-4, 3)$ }
{ $I(4, -3)$ }
\end{ex}

\begin{ex}
Trong mặt phẳng $Oxy$, cho phương trình đường tròn $(x -4)^2   + (y -5)^2 = 16$. Tọa độ tâm của đường tròn là
\choice
{ $I(5, -4)$ }
{ \True $I(4, 5)$ }
{ $I(-5, 4)$ }
{ $I(-4, -5)$ }
\end{ex}

\begin{ex}
Trong mặt phẳng $Oxy$, cho phương trình đường tròn $(x -3)^2   + (y -1)^2 = 4$. Tọa độ tâm của đường tròn là
\choice
{ $I(1, -3)$ }
{ $I(-3, -1)$ }
{ $I(-1, 3)$ }
{ \True $I(3, 1)$ }
\end{ex}

\begin{ex}
Trong mặt phẳng $Oxy$, cho phương trình đường tròn $(x -2)^2   + (y -7)^2 = 25$. Tọa độ tâm của đường tròn là
\choice
{ $I(7, -2)$ }
{ $I(-2, -7)$ }
{ \True $I(2, 7)$ }
{ $I(-7, 2)$ }
\end{ex}

\begin{ex}
Trong mặt phẳng $Oxy$, cho phương trình đường tròn $(x -6)^2   + (y + 6)^2 = 25$. Tọa độ tâm của đường tròn là
\choice
{ $I(6, 6)$ }
{ \True $I(6, -6)$ }
{ $I(-6, 6)$ }
{ $I(-6, -6)$ }
\end{ex}

\begin{ex}
Trong mặt phẳng $Oxy$, cho phương trình đường tròn $(x + 3)^2  + (y + 2)^2 = 16$. Tọa độ tâm của đường tròn là
\choice
{ $I(3, 2)$ }
{ $I(2, -3)$ }
{ $I(-2, 3)$ }
{ \True $I(-3, -2)$ }
\end{ex}

\begin{ex}
Trong mặt phẳng $Oxy$, cho phương trình đường tròn $(x + 1)^2  + (y -6)^2 = 36$. Tọa độ tâm của đường tròn là
\choice
{ \True $I(-1, 6)$ }
{ $I(1, -6)$ }
{ $I(6, 1)$ }
{ $I(-6, -1)$ }
\end{ex}

\begin{ex}
Trong mặt phẳng $Oxy$, cho phương trình đường tròn $(x + 3)^2  + (y + 6)^2 = 9$. Tọa độ tâm của đường tròn là
\choice
{ $I(-6, 3)$ }
{ $I(3, 6)$ }
{ $I(6, -3)$ }
{ \True $I(-3, -6)$ }
\end{ex}


